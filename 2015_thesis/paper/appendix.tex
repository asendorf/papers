We use the following notation. Let $Y_i$ for $i=1,\dots, m$ denote the $d_i \times n$ data
matrix, with each column an observation from the $i$th dataset. Let $Y=[Y_1^H \dots
Y_m^H]^H\in\complex^{d\times n} $ be the entire observation matrix made by stacking $Y_i$
on top of each other. Let $U_i\Sigma_i V_i^H$ be the individual data SVDs of $Y_i$. Let
$\widehat{R} = \frac{1}{n}Y^HY$ be the sample covariance matrix. Defining $U =
\blkdiag(U_1,\dots, U_m)\in\complex^{d\times dm}$, $\Sigma = \blkdiag(\Sigma_1, \dots,
\Sigma_m)\in\complex^{d\times nm}$, $V=[V_1,\dots, V_m]\in\complex^{n\times nm}$, we can
write $\widehat{R}=U\Sigma V^H V \Sigma U^H$. Similarly, define $\widehat{R}_D =
\frac{1}{n}\blkdiag(Y_i^TY_i) = U\Sigma\Sigma^H U^H$. Recall that $x=[x_1^H\dots x_m^H]^H$
is the vector of canonical vectors.

\hiddensection{Problem 1a}
%\subsection{Theory}
\hiddensubsection{Theory}
Our optimization problem is
\begin{equation}\label{eq:1a_prob}
\begin{aligned}
&\argmax_{x_1,\dots,x_m}&&\sum_{i=1}^m\sum_{j=1}^mx_i^HR_{ij}x_j=x^HRx\\
&\text{s.t.} &&x_i^Hx_i  = 1,\,\,\,\,\,\, i=1,\dots,m.
\end{aligned}
\end{equation}
The Lagrangian for this problem is
\begin{equation*}
\begin{aligned}
&L(x,\underline{\lambda}) &&= x^H R x -  \sum_{i=1}^m\lambda_i\left(x_i^Hx_i - 1\right).\\
\end{aligned}
\end{equation*}
Define $\Lambda = \blkdiag(\lambda_1I_{d_1},\dots,\lambda_mI_{d_m})$ to be the matrix with
the Lagrange multipliers on the diagonal. The derivative of the Lagrangian is
\begin{equation*}
\frac{\partial L}{\partial x} = 2R x - 2\Lambda_Dx.
\end{equation*}
Setting the derivative equal to the zero vector results in the following non-normal
generalized eigensystem.
\begin{equation*}
R\widetilde{x} = \Lambda_D\widetilde{x},
\end{equation*}
where $\widetilde{x}$ is a unit norm vector that may be decomposed as $\widetilde{x} =
[\widetilde{x}_1^H,\dots,\widetilde{x}_m^H]^H$ with $\widetilde{x}_i\in\complex^{d_i}$
Therefore, the canonical vectors are
\begin{equation*}
  x = \left[\begin{array}{ccc}\|\widetilde{x}_1\|^{-1} I_{d_1} &0&0\\ 0&\ddots &0 \\
      0&0&\|\widetilde{x}_m\|^{-1}I_{d_m}\\ \end{array}\right]\widetilde{x}.
\end{equation*}
To obtain the canonical correlation, we substitute the canonical vectors into the
objective function. 

\hiddensubsection{Empirical}

As shown in the previous section, the solution to (\ref{eq:1a_prob}) is a non-normal
eigenvalue system. To solve this problem, we use the manopt software package to solve cost
functions on manifolds. The manifold for this problem is the product of $m$ sphere
manifolds constraining the canonical vectors $x_i$ to lie on the $\complex^{d_i}$ unit
sphere. We use the SUMCORR cost function and its gradient
\be
\frac{\partial}{\partial x}= 2Rx
\ee
in the manopt solution.

\hiddensection{Problem 1b}
\hiddensubsection{Theory}
Our optimization problem is
\begin{equation*}
\begin{aligned}
&\argmax_{x_1,\dots,x_m}&&\sum_{i=1}^m\sum_{j=1}^mx_i^HR_{ij}x_j=x^HRx\\
&\text{s.t.} &&x^Hx  = 1.
\end{aligned}
\end{equation*}
The Lagrangian for this problem is
\begin{equation*}
\begin{aligned}
&L(x,\lambda) &&= x^HR x + \lambda(1-x^Hx).\\
\end{aligned}
\end{equation*}
The derivative of the Lagrangian is
\begin{equation*}
\frac{\partial L}{\partial x} = 2Rx - 2x.
\end{equation*}
Setting the derivative equal to the zero vector results in the following eigensystem.
\begin{equation}\label{eq:prob1b}
  Rx = \lambda x
\end{equation}
From this relationship, if we substitute this solution into the objective function, we
obtain
\begin{equation}\label{eq:rho1b}
  \rho = x^H R x = x^H(\lambda x) = \lambda.
\end{equation}

\hiddensubsection{Empirical}
We plug in $\widehat{R}$ into (\ref{eq:prob1b}) for $R$ and solve the eigenvalue
decomposition. The eigenvector $x$ is the canonical vector and the eigenvalue $\lambda$ is
the canonical correlation. This problem is typically ill-posed as the maximum solution is
typically found by setting only one $x_i$ to be nonzero corresponding to the $R_{ii}$ with
the largest variance. We advise to not use this formulation of MCCA.

\hiddensection{Problem 1c}
\hiddensubsection{Theory}
Our optimization problem is
\begin{equation*}
\begin{aligned}
&\argmax_{x_1,\dots,x_m}&&\sum_{i=1}^m\sum_{j=1}^mx_i^HR_{ij}x_j=x^HRx\\
&\text{s.t.} &&x_i^HR_{ii}x_i  = 1.
\end{aligned}
\end{equation*}
The Lagrangian for this problem is
\begin{equation*}
\begin{aligned}
&L(x,\underline{\lambda}) &&= x^HR x - \sum_{i=1}^m\lambda_i\left(x_i^HR_{ii}x_i\right).\\
\end{aligned}
\end{equation*}
Define $\Lambda_D\in\complex^{d \times d}=\blkdiag(\lambda_1I_{d_1},\dots,\lambda_mI_{d_m})$. The derivative of the Lagrangian is
\begin{equation*}
\frac{\partial L}{\partial x} = 2R x - 2\Lambda_DR_Dx.
\end{equation*}
Setting the derivative equal to the zero vector results in the non-normal generalized eigensystem.
\be\ba
&R\widetilde{x} &&= \Lambda_DR_D\widetilde{x}.\\
\ea\ee
To obtain the canonical vectors, we make the transformation
\be
x_i = \frac{R_{ii}^{-1/2}\widetilde{x}_i}{\|\widetilde{x}_i\|}.
\ee

\hiddensubsection{Empirical}

Making the transformation 
\be
x_i = R_{ii}^{-1/2}\widetilde{x}_i,
\ee
our optimization problem becomes
\begin{equation*}
\begin{aligned}
&\argmax_{\widetilde{x}}&&\widetilde{x}^HR_D^{-1/2}RR_D^{-1/2}\widetilde{x}\\
&\text{s.t.} &&\widetilde{x}_i^H\widetilde{x}_i = 1.\\
\end{aligned}
\end{equation*}

As shown in the previous section, the solution to this problem is a non-normal eigenvalue
system. To solve the above problem, we use the manopt software package to solve cost
functions on manifolds. The manifold for this problem is the product of $m$ sphere
manifolds constraining each canonical vector to lie on the $\complex^{d_i}$ unit sphere.
We use the SUMCORR cost function and the derivative
\be
\frac{\partial}{\partial \widetilde{x}}= 2R_D^{-1/2}RR_D^{-1/2}\widetilde{x}.
\ee
 We substitute the empirical sample covariances 
\be
\widehat{R} = \frac{1}{n}YY^T,\,\,\, \widehat{R}_D = \frac{1}{n}\blkdiag(Y_iY_i^T)
\ee
 for the unknown $R$ and $R_D$ in the cost and gradient functions. To obtain the canonical
 vectors $x_i$, we make the transformation
\be
x_i = R_{ii}^{-1/2}\widetilde{x}_i.
\ee

Using our notation for $R$ and $R_D$ from the data SVDs, we have
\be
R_D^{-1/2}RR_D^{-1/2} = UV^HVU^H
\ee
and
\be
x = U\Sigma^{-1}U^H \widetilde{x}.
\ee
The canonical correlation is
\be
\widehat{\rho} = x^HR_D^{-1/2}RR_D^{-1/2}x = \widetilde{x}^HUV^HVU^H\widetilde{x}.
\ee

\hiddensection{Problem 1d}
\hiddensubsection{Theory}
Our optimization problem is
\begin{equation*}
\begin{aligned}
&\argmax_{x_1,\dots,x_m}&&\sum_{i=1}^m\sum_{j=1}^mx_i^HR_{ij}x_j=x^H Rx\\
&\text{s.t.} &&x^HR_D x  = 1.\\
\end{aligned}
\end{equation*}
The Lagrangian for this problem is
\begin{equation*}
L(x,\lambda) = x^HR x + \lambda(1-x^HR_Dx).
\end{equation*}
The derivative of the Lagrangian is
\begin{equation*}
\frac{\partial L}{\partial x} = 2Rx - 2\lambda R_Dx.
\end{equation*}
Setting the derivative equal to the zero vector results in the following generalized eigensystem.
\begin{equation*}
R_D^{-1}Rx = \lambda x.
\end{equation*}
Let $\widetilde{x}= R_D^{1/2}x$  so that the eigensystem becomes
\be
R_D^{-1/2}R R_D^{-1/2}\widetilde{x}=\lambda \widetilde{x}
\ee
where $\|\widetilde{x}\|_2=1$. The canonical vectors are $x=R_D^{-1/2}\widetilde{x}$ and
the canonical correlation is $\rho = \lambda$.

\hiddensubsection{Empirical}
Our empirical eigen-problem is
$\widehat{R}_D^{-1/2}\widehat{R}\widehat{R}_D^{-1/2}\widetilde{x}=\lambda
\widetilde{x}$. Using data SVDs, 
\be
\widehat{R}_D^{-1/2}\widehat{R}\widehat{R}_D^{-1/2}= UV^HVU^H.\\
\ee
Let $Q\Lambda Q^H$ be the eigenvalue decomposition of $UV^H VU^H$.  To obtain canonical
vectors consistent with the the constraint function, we make the transformation
\be
x = U\Sigma^{-1}U^HQ.
\ee
Substituting this expression into the objective function, we obtain
\be
\widehat{\rho} = \lambda.
\ee

\hiddensection{Problem 2a}

\hiddensubsection{Theory}
Our optimization problem is
\begin{equation*}
\begin{aligned}
&\argmax_{x}&&\sum_{i=1}^m\sum_{j=1}^m(x_i^HR_{ij}x_j)^2=\|X^HRX\|_F^2\\
&\text{s.t.} &&x_i^Hx_i = 1, i=1,\dots,m.\\
\end{aligned}
\end{equation*}

To calculate the gradient of the cost function, we use the double summation version of the
cost function. We have that
\beq\label{eq:2a_grad}\ba
&\frac{\partial}{\partial x_i} &&= 4\sum_{j=1}^m (x_i^H R_{ij}x_j) R_{ij}x_j\\
&&& = R_{i,:} X (X^HR X)_{:,i}\\
\ea\eeq
where 
\be
R_{i,:} = [R_{i,1},\dots, R_{i,m}],\,\,\,\, (X^H RX)_{:,i} = [x_1^HR_{1,i}x_i, \dots, x_m^H R_{m,i}x_i]^H.
\ee
Thus
\be
\frac{\partial}{\partial x} = \left[\begin{array}{c} \frac{\partial}{\partial x_1}\\ \vdots \\
    \frac{\partial}{\partial x_m}\end{array}\right].
\ee
If we try to use a Lagrangian method to solve this problem, we end up with an eigenvalue
problem of the form $\widetilde{R}(x)x = \Lambda_D x$. As the matrix $\widetilde{R}(x)$ is
dependent on the eigenvector $x$ and $\Lambda_D= \diag(\lambda_1 I_{d1},\dots,\lambda_m
I_{dm})$, this is a highly non regular eigenvalue problem. 

\hiddensubsection{Empirical}

To solve the problem above, we use the manopt software package to solve cost functions on
manifolds. Each of our canonical vectors are constrained on the $d_i$ unit sphere. We use
the SSQCORR cost function and the derivative in (\ref{eq:2a_grad}). We substitute the empirical
sample covariance 
\be
\widehat{R} = \frac{1}{n}YY^T
\ee
for the unknown $R$ in the cost and gradient functions.

\hiddensection{Problem 2b}

\hiddensubsection{Theory}
Our optimization problem is
\begin{equation*}
\begin{aligned}
&\argmax_{x}&&\sum_{i=1}^m\sum_{j=1}^m(x_i^HR_{ij}x_j)^2\\
&\text{s.t.} &&x^Hx  = 1\\
\end{aligned}
\end{equation*}

The derivative of our cost function is the same as in (\ref{eq:2a_grad}). If we try to use
a Lagrangian method to solve this problem, we end up with an eigenvalue problem of the
form $\widetilde{R}(x)x = \lambda x$. As the matrix $\widetilde{R}(x)$ is dependent on the
eigenvector $x$, this is a highly non regular eigenvalue problem. 

\hiddensubsection{Empirical}

To solve the problem above, we use the manopt software package to solve cost functions on
manifolds. Our manifold is simpler as we only have one constraint that $x^Hx=1$. We use
the SSQCORR cost function and the derivative in (\ref{eq:2a_grad}). We substitute the empirical
sample covariance 
\be
\widehat{R} = \frac{1}{n}YY^T
\ee
for the unknown $R$ in the cost and gradient functions. However, the solution to this
problem will typically set the only one $x_i$ to be non-zero corresponding to the $R_{ii}$
with the largest eigenvalue. We advise not to use this formulation of MCCA.


\hiddensection{Problem 2c}

\hiddensubsection{Theory}
Our optimization problem is
\begin{equation*}
\begin{aligned}
&\argmax_{x}&&\sum_{i=1}^m\sum_{j=1}^m(x_i^HR_{ij}x_j)^2\\
&\text{s.t.} &&x_i^HR_{ii}x_i=1.\\
\end{aligned}
\end{equation*}

The derivative of our cost function is the same as in (\ref{eq:2a_grad}). If we try to use
a Lagrangian method to solve this problem, we end up with an eigenvalue problem of the
form $\widetilde{R}(x)x = \Lambda_D R_D x$. As the matrix $\widetilde{R}(x)$ is dependent
on the eigenvector $x$ and and $\Lambda_D= \diag(\lambda_1 I_{d1},\dots,\lambda_m
I_{dm})$, this is a highly non regular eigenvalue problem. 


\hiddensubsection{Empirical}

We first make the transformation 
\be
x_i = R_{ii}^{-1/2}\widetilde{x}_i. 
\ee
Our optimization problem becomes
\begin{equation*}
\begin{aligned}
&\argmax_{\widetilde{x}}&&\|\widetilde{X}^HR_D^{-1/2}RR_D^{-1/2}\widetilde{X}\|_F^2\\
&\text{s.t.} &&\widetilde{x}_i^H\widetilde{x}_i = 1, i=1,\dots,m\\
\end{aligned}
\end{equation*}
This is the same type of optimization problem as Problem 2a if we replace $R$ with
$R_D^{-1/2}RR_D^{-1/2}$.

To solve this problem above, we use the manopt software package to solve cost functions on
manifolds. Our manifold consists of $m$ constraints, $\widetilde{x}^H_i\widetilde{x}_i=1$,
that is $m$ vectors constrained on the $d_i$ unit sphere. We use the SSQCORR cost function
and the derivative in (\ref{eq:2a_grad}). We substitute the empirical sample covariances 
\be
\widehat{R} = \frac{1}{n}YY^T,\,\,\, \widehat{R}_D = \frac{1}{n}\blkdiag(Y_iY_i^T)
\ee
 for the unknown $R$ and $R_D$ in the cost and gradient functions.

To obtain the canonical vectors $x_i$, we make the transformation
\be
x_i = R_{ii}^{-1/2}\widetilde{x}_i.
\ee

Using our notation for $R$ and $R_D$ from the data SVDs, we have
\be
R_D^{-1/2}RR_D^{-1/2} = UV^HVU^H
\ee
and
\be
x = U\Sigma^{-1}U^H \widetilde{x}.
\ee


\hiddensection{Problem 2d}

\hiddensubsection{Theory}
Our optimization problem is
\begin{equation*}
\begin{aligned}
&\argmax_{x}&&\sum_{i=1}^m\sum_{j=1}^m(x_i^HR_{ij}x_j)^2\\
&\text{s.t.} &&x^HR_Dx  = m.\\
\end{aligned}
\end{equation*}

The derivative of our cost function is the same as in (\ref{eq:2a_grad}). If we try to use
a Lagrangian method to solve this problem, we end up with an eigenvalue problem of the
form $\widetilde{R}(x)x = \lambda R_D x$. As the matrix $\widetilde{R}(x)$ is dependent on
the eigenvector $x$ and, this is a highly non regular eigenvalue problem. 

\hiddensubsection{Empirical}

We first make the transformation 
\be
x_i = R_{ii}^{1/2}\widetilde{x}_i. 
\ee
Our optimization problem becomes
\begin{equation*}
\begin{aligned}
&\argmax_{\widetilde{x}}&&\|\widetilde{X}^HR_D^{-1/2}RR_D^{-1/2}\widetilde{X}\|_F^2\\
&\text{s.t.} &&\widetilde{x}^H\widetilde{x} = 1\\
\end{aligned}
\end{equation*}
This is the same type of optimization problem as Problem 2b if we replace $R$ with
$R_D^{-1/2}RR_D^{-1/2}$.

To solve this problem above, we use the manopt software package to solve cost functions on
manifolds. Our manifold consists of only one constraint, $\widetilde{x}^H\widetilde{x} =
1$, which is a vector constrained on the $\reals^d$ unit sphere. We use the SSQCORR cost function
and the derivative in (\ref{eq:2a_grad}). We substitute the empirical sample covariances 
\be
\widehat{R} = \frac{1}{n}YY^T,\,\,\, \widehat{R}_D = \frac{1}{n}\blkdiag(Y_iY_i^T)
\ee
 for the unknown $R$ and $R_D$ in the cost and gradient functions.

To obtain the canonical vectors $x_i$, we make the transformation
\be
x = R_d^{-1/2}\widetilde{x}
\ee

Using our notation for $R$ and $R_D$ from the data SVDs, we have
\be
\widetilde{X}^HR_D^{-1/2}RR_D^{-1/2} = UV^HVU^H
\ee
and
\be
x = U\Sigma^{-1}U^H \widetilde{x}.
\ee

\hiddensection{Problem 3a}
\hiddensubsection{Theory}
Our optimization problem is
\begin{equation*}
\begin{aligned}
&\argmax_{x}&&\lambda_1\\
&\text{s.t.} && x_i^Hx_i = 1, i=1,\dots,m\\
&&& \Phi(x) a = \lambda_1 a\\
&&& a^Ha=1.\\
\end{aligned}
\end{equation*}
We may write $\Phi(x) = X^HRX$. Using this fact and third constraint of this optimization,
the second constraint may be written as $a^HX^HR Xa=\lambda_1$. Define
$\widetilde{a}=Xa$. As a consequence of the first constraint function,
\begin{equation*}
\|\widetilde{a}\|^2=a^HX^HXa = a^Ha=1.
\end{equation*}
Our modified optimization problem is
\begin{equation*}
\begin{aligned}
&\argmax_{\tilde{a}}&&\lambda_1\\
&\text{s.t}&& \tilde{a}^HR\tilde{a} = \lambda_1\\
&&& \widetilde{a}^H\widetilde{a}=1.\\
\end{aligned}
\end{equation*}
Therefore, $\widetilde{a}$ is the unit norm eigenvector corresponding to the largest
eigenvalue of $R$. To solve for the canonical coefficients, we have $\widetilde{a} = Xa$
which implies $x_i=\frac{\widetilde{a}_i}{a_i}$. As $a_i$ is a scalar, and $x_i$ is
required to have unit norm, we have that
$x_i=\frac{\widetilde{a}_i}{\|\widetilde{a}_i\|}$. This implies
$x=\Lambda_{\widetilde{a}}^{-1}\widetilde{a}$ where
$\Lambda_{\widetilde{a}}\in\complex^{d\times d}=\blkdiag(\|\widetilde{a}_i\|I_{d_i})$. The
canonical correlation is simply $\rho = \lambda_1$.

\hiddensubsection{Empirical}
Our empirical eigen-system is $\widehat{R}\widetilde{a}=\lambda_1\widetilde{a}$ where
$\widehat{R}= \frac{1}{n}YY^H$ is the sample covariance matrix. Let $Q\Lambda Q^H$ be
the eigenvalue decomposition of $\widehat{R}$. Let $q$ be the leftmost column of $Q$ and
decomposed as $q^H=[q_1^H,\dots,q_m^H]$ with $q_i\in\complex^{d_i}$. Then
\be\ba
& \widehat{\rho} = \lambda_1\\
& \widehat{x}= \Lambda_{\widetilde{q}}^{-1}q\\
\ea
\ee
where $\Lambda_{\widetilde{q}}\in\complex^{d\times d}=\blkdiag(\|\widetilde{q}_i\|I_{d_i})$.

\hiddensection{Problem 3b}

\hiddensubsection{Theory}
Our optimization problem is
\begin{equation*}
\begin{aligned}
&\argmax_{x}&&\lambda\\
&\text{s.t.} &&x^Hx  = 1\\
&&& \Phi(x) a = \lambda a\\
&&& a^Ha=1.\\
\end{aligned}
\end{equation*}
We may write $\Phi(x) = X^HR X$. Using this fact and third constraint of this
optimization, the second constraint may be written as $a^HX^H R Xa=\lambda$.

Let $R=U\Sigma V^HV\Sigma^H U^H$ be a decomposition of $R$ using the block SVDs of the
individual covariance matrices $R_{ii}$. Let $\widetilde{a}=Xa$. We wish to maximize
$\lambda=\widetilde{a}^H R\widetilde{a}$, with $\|\widetilde{a}\|=1$. This is equivalent
to
\be\ba
&\argmax_{\widetilde{a}}&&\|R^{1/2}\widetilde{a}\|_2\\
&\text{s.t.} &&\|\widetilde{a}\|=1.
\ea\ee
Now
\be\ba
&\|R^{1/2}\widetilde{a}\|_2 && = \|P\Sigma U^H\widetilde{a}\|_2\\
\ea\ee
where $P\in\complex^{d\times d}$ is composed of sub-matrices $P_{ij}\in\complex^{d_i\times
  d_j}=\text{corr}(y_i,y_j)$. Note that $P_{ii}=I_{d_i}$. The entries of $P$ are all
between $-1$ and $1$.  Now since $U$ is an orthonormal matrix and the largest entries in
$P$ have norm 1, to maximize this norm, $\widetilde{a}$ should be the column of $U$
corresponding to the largest value in $\Sigma$. Since $U$ is block diagonal,
$\widetilde{a}=[0^H\dots 0^H u_{i1}^H 0^H]^H$ where $u_{i1}$ is the leftmost left singular
vector of $R_{ii}$ where $i$ is the dataset with the largest singular value. Therefore,
$\rho=\widetilde{a}^HU\Sigma
PP^T\Sigma^HU^H\widetilde{a}=\sigma_{i1}^2P_{ii}=\sigma_{i1}^2$ as $P_{ii}=1$. Therefore,
the canonical vectors are
\be
x_i = \begin{cases} u_{i1} & \text{dataset } i \text{ has largest singular value}\\
0 & \text{otherwise}\\ \end{cases}
\ee
This is obviously undesirable as all but one canonical vector is 0. We advise to not use
this formulation of MCCA.

\hiddensubsection{Empirical}
In the empirical setting, we substitute $\widehat{R}$ as the sample covariance
estimate. Recall that $\widehat{R} = U\Sigma V^H V \Sigma^HU^H$. Letting $\widetilde{a} =
Xa$, our optimization problem is
\be\ba
&\argmax_{\widetilde{a}}&&\|R^{1/2}\widetilde{a}\|_2\\
&\text{s.t.} &&\|\widetilde{a}\|=1
\ea\ee
We can rewrite this as
\be\ba
&\|R^{1/2}\widetilde{a}\|_2 && = \|V\Sigma U^H\widetilde{a}\|_2.\\
\ea\ee
Now since $U$ is an orthogonal matrix and the columns of $V$ are unit norm, to maximize
this norm, $\widetilde{a}$ should be the column of $U$ corresponding to the largest value
in $\Sigma$. Since $U$ is block diagonal, $\widetilde{a}=[0^H\dots 0^H u_{i1}^H 0^H]^H$
where $u_{i1}$ is the leftmost left singular vector of $R_{ii}$ where $i$ is the dataset
whose sample covariance matrix has the largest singular value. The value of
$\widehat{\rho}$ is the value of the largest singular value squared. This formulation of MCCA
results in canonical vectors that are 0 for all but one dataset. This obviously is very
undesirable and we advise to not use this formulation for MCCA.

\hiddensection{Problem 3c}

\hiddensubsection{Theory}
Our optimization problem is
\begin{equation*}
\begin{aligned}
&\argmax_{x}&&\lambda\\
&\text{s.t.} &&x_i^HR_{ii}x_i  = 1\,\,,1\leq i\leq m\\
&&& \Phi(x) a = \lambda a\\
&&& a^Ha=1.\\
\end{aligned}
\end{equation*}
We may write $\Phi(x) = X^H R X$. Using this fact and third constraint of this
optimization, the second constraint may be written as $a^HX^H R Xa=\lambda$. If we assume
that $R_D$ is positive definite (which requires it to be full rank), we can rewrite this
as $a^HX^HR_D^{1/2}R_D^{-1/2}RR_D^{-1/2}R_D^{1/2}Xa=\lambda$. Let
$\widetilde{a}=R_D^{1/2}Xa$. Now by the first and third constraints
\begin{equation*}
\|\tilde{a}\|^2 = a^HX^HR_DXa=a^H I_m a = a^Ha = 1.
\end{equation*}
Our modified optimization problem is
\begin{equation*}
\begin{aligned}
&\argmax_{\widetilde{a}}&&\lambda\\
&\text{s.t}&& \widetilde{a}^HR_D^{-1/2}RR_D^{-1/2}\widetilde{a} = \lambda\\
&&& \widetilde{a}^H\widetilde{a}=1.\\
\end{aligned}
\end{equation*}
Therefore, $\widetilde{a}$ is the eigenvector corresponding to the largest eigenvalue of
$R_D^{-1/2}RR_D^{-1/2}$. To solve for our original canonical coefficients, recall that
$\widetilde{a}=R_D^{1/2}Xa$. As $R_D$ and $X$ are block diagonal, we have
$\widetilde{a}_i=R_{ii}^{1/2}x_ia_i$, implying $x_i =
\frac{1}{a_i}R_{ii}^{-1/2}\widetilde{a}_i$. By the first constraint, 
\be
x_i^HR_{ii}x_i  = \frac{\widetilde{a}_i^H\widetilde{a}_i}{a_i^2} =1.
\ee
Letting $a_i = \|\widetilde{a}_i\|$ satisfies this constraint. Therfore, the canonical vector is
\be
x_i = \frac{R_{ii}^{-1/2}\widetilde{a}_i}{\|\widetilde{a}_i\|}.
\ee
Thus
\be
x = \Lambda_{\widetilde{a}}^{-1}R_D^{-1/2}\widetilde{a}
\ee
where $\Lambda_{\widetilde{a}}\in\complex^{d\times d}=\blkdiag(\|\widetilde{a}_i\|I_{d_i})$.

%This seems related to the SUMCORR problems.
%\begin{equation*}
%\begin{aligned}
%&\Sigma_D^{-1/2}\Sigma\Sigma_D^{-1/2}\tilde{a}&&=\lambda\tilde{a}\\
%&\Sigma_D^{-1/2}\Sigma\Sigma_D^{-1/2}(\Sigma_D^{1/2}\Lambda_{\tilde{a}}w)&&=\lambda\Sigma_D^{1/2}\Lambda_{\tilde{%a}}w\\
%&\Sigma_D^{-1/2}\Sigma\Lambda_{\tilde{a}}w&&=\lambda\Sigma_D^{1/2}\Lambda_{\tilde{a}}w\\
%&\Sigma\Lambda_{\tilde{a}}w&&=\lambda\Sigma_D\Lambda_{\tilde{a}}w\\
%\end{aligned}
%\end{equation*}

\hiddensubsection{Empirical}
Our empirical eigen-system is
$\widehat{R}_D^{-1/2}\widehat{R}\widehat{R}_D^{-1/2}\widetilde{a} =
\widehat{\rho}\widetilde{a}$. Using the SVD notation for our empirical data matrices, we
have that
\begin{equation*}
\begin{aligned}
&\widehat{R}_D^{-1/2}\widehat{R}\widehat{R}_D^{-1/2}&&=\left(U\Sigma\Sigma^HU^H\right)^{-1/2}\left(U\Sigma V^HV\Sigma^H U^H\right)\left(U\Sigma\Sigma^HU^H\right)^{-1/2}\\
&&&=U(\Sigma\Sigma^H)^{-1/2}U^HU\Sigma V^HV\Sigma^HU^HU(\Sigma\Sigma^H)^{-1/2}U^H\\
&&& = U(\Sigma\Sigma^H)^{-1/2}\Sigma V^HV\Sigma^H(\Sigma\Sigma^H)^{-1/2}U^H\\
&&&=U\widetilde{V}^H\widetilde{V}U^H
\end{aligned}
\end{equation*}
where $\tilde{V}\in\complex^{n\times d}=[V_1(:,1:d_1),\dots,V_m(:,1:d_m)]$. Defining $\widehat{C} = \widetilde{V}^H\widetilde{V}$ and its eigenvalue decomposition $\widehat{C} = \widehat{F}\widehat{K}\widehat{F}^H$, then we have that the MCCA empirical solution is
\be\ba
& \widehat{\rho} = \widehat{k}_1\\
& \widehat{x} = U \widetilde{\Sigma}^{-1}\Lambda_{\widehat{f}_1}^{-1}\widehat{f}_1
\ea\ee
where $\widetilde{\Sigma} = \blkdiag\left(\Sigma_1(1:d_1,1:d_1),\dots, \Sigma_m(1:d_m,1:d_m)\right)$.

\hiddensection{Problem 3d}
\hiddensubsection{Theory}
We proceed very similarly as above. Our optimization problem is
\begin{equation*}
\begin{aligned}
&\argmax_{x}&&\lambda\\
&\text{s.t.} &&xR_Dx  = 1\\
&&& \Phi(x) a = \lambda a\\
&&& a^Ha=1.\\
\end{aligned}
\end{equation*}
Substituting  $\widetilde{x} = R_D^{1/2}x$ into the above problem yields
\begin{equation*}
\begin{aligned}
&\argmax_{x}&&\lambda\\
&\text{s.t.} &&\widetilde{x}^H\widetilde{x}= 1\\
&&& \widetilde{X}^HR_D^{-1/2}RR_d^{-1/2}\widetilde{X} a = \lambda a\\
&&& a^Ha=1.\\
\end{aligned}
\end{equation*}
This is now the same problem as 3b except we replace $R$ with
$R_D^{-1/2}RR_D^{-1/2}$. Using the SVD notation as in 3b, we have that
$R_D^{-1/2}RR_D^{-1/2}=UP^TPU^H$. Recall that the diagonals of $P$ are 1 and that every
entry of $P$ has a norm of no greater than 1. We can clearly see that this problem does
not have a unique solution. We can set any $x_i=u_i/\sigma_i$ where $u_i$ is any left singular
vector or $R_{ii}$ corresponding to the singular value $\sigma_i$. We then set all other
$x_i=0$. Choosing canonical vectors in this fashion results in $\rho=1$. This solution is
non-unique and clearly undesirable. Therefore, the canonical vectors are
\be
x_i = \begin{cases} u_{i}/\sigma_i & \text{for one dataset}\\
0 & \text{for all others}\\ \end{cases}.
\ee
We advise to not use this formulation of MCCA.

\hiddensubsection{Empirical}
The solutions to this problem are not unique. Take the data SVD of one dataset $Y_i$ and
set $x_i = u_i/\sigma_i$ and all others equal to 0. 

\hiddensection{Problem 4a}

The optimization problem is 
\begin{equation*}
\begin{aligned}
&\argmin_{x}&&\lambda\\
&\text{s.t.} &&x_i^Hx_i  = 1, i=1,\dots,m\\
&&& \Phi(x) a = \lambda a\\
&&& a^Ha=1.\\
\end{aligned}
\end{equation*}

Here we proceed exactly as in problem 3a except that we choose the eigenvector
corresponding to the smallest, (potentially zero) eigenvalue.

\hiddensection{Problem 4b}
\hiddensubsection{Theory}
The optimization problem is 
\begin{equation*}
\begin{aligned}
&\argmin_{x}&&\lambda\\
&\text{s.t.} &&x^Hx  = 1\\
&&& \Phi(x) a = \lambda a\\
&&& a^Ha=1.\\
\end{aligned}
\end{equation*}
Choosing the canonical vectors the same way as in 3b makes $\Phi(x)$ singular. Therefore
we can achieve an eigenvalue of 0. This is optimal as $\Phi(x)$ is positive
semi-definite. This solution is not unique and undesirable. 

\hiddensubsection{Empirical}
The solutions to this problem are not unique. Take the data SVD of one dataset $Y_i$ and
set $x_i = u_i/\sigma_i$ and all others equal to 0 for any dataset and any singular
vector/value pair.

\hiddensection{Problem 4c}

The optimization problem is 
\begin{equation*}
\begin{aligned}
&\argmin_{x}&&\lambda\\
&\text{s.t.} &&x_i^HR_{ii}x_i  = 1, i=1,\dots,m\\
&&& \Phi(x) a = \lambda a\\
&&& a^Ha=1.\\
\end{aligned}
\end{equation*}

Here we proceed exactly as in problem 3c except that we choose the eigenvector
corresponding to the smallest, nonzero eigenvalue.

\hiddensection{Problem 4d}

\hiddensubsection{Theory}

The optimization problem is 
\begin{equation*}
\begin{aligned}
&\argmin_{x}&&\lambda\\
&\text{s.t.} &&x^HR_Dx  = 1\\
&&& \Phi(x) a = \lambda a\\
&&& a^Ha=1.\\
\end{aligned}
\end{equation*}
Choosing the canonical vectors the same way as in 3d makes $\Phi(x)$ singular. Therefore
we can achieve an eigenvalue of 0. This is optimal as $\Phi(x)$ is positive
semi-definite. This solution is not unique and undesirable. 

\hiddensubsection{Empirical}
The solutions to this problem are not unique. Take the data SVD of one dataset $Y_i$ and
set $x_i = u_i/\sigma_i$ and all others equal to 0 for any dataset and any singular
vector/value pair.

\hiddensection{Problems 5a-d Theory}

The GENVAR problem does not offer a closed form solution. To solve these problems we use
the manopt software package. The cost function is
\beq\label{eq:genvar_cost}
|X^HRX|
\eeq
where $X=\blkdiag(x_1,\dots,x_m)$. The gradient with respect to the matrix $X$ is
\be
\frac{\partial}{\partial X} = 2|X^H RX|RX(X^HRX)^{-1}.
\ee
Let $\ones_{d_i}\in\complex^{d_i}$ be the vector of all ones. Let
$A=\blkdiag(\ones_{d_1},\dots,\ones_{d_m})$. Then the gradient with respect to the vector
$x$ can be extracted via
\beq\label{eq:genvar_grad}
\frac{\partial}{\partial x} = 2|X^H RX|RX(X^HRX)^{-1}\odot A
\eeq
where $\odot$ represents element-wise multiplication. Choosing the appropriate cost
function manifolds completes the solution using manopt as we see below in the empirical
versions. 

\hiddensection{Problem 5a Empirical}
The canonical vectors are each constrained on the $\complex^{d_i}$ unit sphere. The
manifold for the problem is the product of $m$ of these sphere manifolds. We use the
sample covariance matrix $\widehat{R}$ for the unknown $R$ in (\ref{eq:genvar_cost}) and
(\ref{eq:genvar_grad}). 

\hiddensection{Problem 5b Empirical}
The canonical vectors are each constrained on the $\complex^{d}$ unit sphere. The
manifold for the problem is therefore one sphere manifolds. We use the
sample covariance matrix $\widehat{R}$ for the unknown $R$ in (\ref{eq:genvar_cost}) and
(\ref{eq:genvar_grad}). 

\hiddensection{Problem 5c Empirical}
The constraints for this problem are $x_i^HR_{ii}x_i = 1$ for $i=1,\dots,m$. Here we make
the transformation
\be
\widetilde{x} = R_{ii}^{1/2}x
\ee
which results in the constraints $\widetilde{x}^H_i\widetilde{x}_i = 1$ for
$i=1,\dots,m$. The cost function becomes
\be
|X^HRX| = |\widetilde{X}^HR_D^{-1/2}RR_D^{-1/2}\widetilde{X}|.
\ee
We see that this is the same type of problem as 5a with $\widetilde{x}$ replacing $x$ and
$R_D^{-1/2}RR_D^{-1/2}$ replacing $R$. We make this substitution and use the sample
covariance matrices $\widehat{R}$ and $\widehat{R}_D$ in (\ref{eq:genvar_cost}) and
(\ref{eq:genvar_grad}). The manifold for this problem is the product of $m$
$\complex^{d_i}$ sphere manifolds.

\hiddensection{Problem 5d Empirical}
The single constraint for this problem is $x^HR_Dx = 1$. Here we make
the transformation
\be
\widetilde{x} = R_{ii}^{1/2}x
\ee
which results in the constraint $\widetilde{x}^H\widetilde{x} = 1$ for
$i=1,\dots,m$. The cost function becomes
\be
|X^HRX| = |\widetilde{X}^HR_D^{-1/2}RR_D^{-1/2}\widetilde{X}|.
\ee
We see that this is the same type of problem as 5a with $\widetilde{x}$ replacing $x$ and
$R_D^{-1/2}RR_D^{-1/2}$ replacing $R$. We make this substitution and use the sample
covariance matrices $\widehat{R}$ and $\widehat{R}_D$ in (\ref{eq:genvar_cost}) and
(\ref{eq:genvar_grad}). The manifold for this problem is one $\complex^{d}$ sphere manifold.
