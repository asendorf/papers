In this section we provide the data model that we will use throughout the chapter. This
model emulates the linear signal-plus-noise model used in many applications. We then
provide the optimization problems and solutions to canonical correlation analysis (CCA),
regularized CCA (RCCA), informative CCA (ICCA), and limit RCCA (LRCCA). 

\subsection{Data Model}

Let $\widetilde{X}_n=\left[\widetilde{x}_1,\dots,\widetilde{x}_n\right]$ and
$\widetilde{Y}_n=\left[\widetilde{y}_1,\dots,\widetilde{y}_n\right]$ be two datasets with
observations $\widetilde{x}_i\in\reals^{p}$ and
$\widetilde{y}_i\in\reals^{q}$. Throughout, we model the datasets as
\beq\label{eq:model}\ba
&\widetilde{X}_n = U_x\Theta_xV_x^T + X_n,\\
&\widetilde{Y}_n = U_y\Theta_yV_y^T + Y_n\\
\ea\eeq where $U_x\in\reals^{p\times r}$, $U_y\in\reals^{q\times r}$ are independent
orthonormal matrices, $V_x\in\reals^{n\times r}$ and $V_y\in\reals^{n\times r}$ are
orthogonal matrices such that $\E{V_x^TV_y}=P=\diag\left(\rho_1,\dots,\rho_r\right)$ with
$0\leq\rho_i\leq1$, and $\Theta_x=\diag(\theta_{x1},\dots,\theta_{xr})$ and
$\Theta_y=\diag(\theta_{y1},\dots,\theta_{yr})$. We denote $p$ as the dimension of the
first dataset, $q$ as the dimension of the second dataset, $n$ as the number of training
samples, and $r$ as the maximum number of signals in either
dataset. $X_n\in\reals^{p\times n}$ and $Y_n\in\reals^{p\times n}$ model the system noise
and are assumed to be independent. In this regard, our model accounts for different
dimensional signal subspaces in $\widetilde{X}$ and $\widetilde{Y}$ by setting the
appropriate $\theta$ and $\rho$ to zero. Finally, define
$\Rxxhat=\frac{1}{n}\widetilde{X}_n\widetilde{X}_n^T$,
$\Ryyhat=\frac{1}{n}\widetilde{Y}_n\widetilde{Y}_n^T$, and
$\Rxyhat=\frac{1}{n}\widetilde{X}_n\widetilde{Y}_n^T$ as the sample covariance matrices.

\subsection{Empirical CCA}

The goal of CCA is to find a linear transformation for each dataset that maximizes the
correlation between the datasets in the projected spaces. We represent the linear
transformations with the canonical vectors $w_x\in\reals^{p\times 1}$ and
$w_y\in\reals^{q\times 1}$ and the projection with the canonical variates $z_x=w_x^Hx$ and
$z_y=w_y^Hy$. The objective is to find the canonical vectors $w_x$ and $w_y$ that
maximize the correlation between the canonical variates $z_x$ and $z_y$. Formally, the
optimization problem is
\begin{equation}\label{eq:cca_opt}
  \begin{aligned}
    &\argmax_{w_x,w_y}&&\rho = \E{z_xz_y}\\
    & \text{subject to}&&\E{z_x^2}=1, \E{z_y^2}=1.
  \end{aligned}
\end{equation}
We may obtain a closed form solution for (\ref{eq:cca_opt}) through the SVD of
\be
\Cccahat=\Rxxhat^{-1/2}\Rxyhat\Ryyhat^{-H/2},
\ee
which relies on estimates of the unknown covariance matrices. Let $FKG^H$ be the SVD of
$\Cccahat$ where 
$F=[f_1,\dots,f_{p}]$, $K=\diag(k_1,\dots,k_{\min\left(p,q\right)})$, and
$G=[g_1,\dots,g_{q}]$. Then the solution for the canonical vector pair corresponding to
the largest canonical correlation is
\beq\label{eq:cca_svd_sol}\ba 
&\rho = k_1\\ 
&w_x = \Rxxhat^{-1/2}f_1\\ 
&w_y = \Ryyhat^{-1/2}g_1.\\ 
\ea\eeq
To find higher order canonical vector and correlation pairs we take successive singular
vector and value pairs of $\Cccahat$.

\subsection{RCCA}

When $n<p+q$, CCA reports a perfect correlation of $\rho=1$ regardless of the true
correlation \cite{pezeshki2004empirical}. To overcome this performance loss, RCCA
introduces a regularization parameter, $\eta$, that adds a multiple of the identity matrix
to the sample covariance matrix of each dataset. Formally, the RCCA optimization problem is
\begin{equation}\label{eq:rcca_opt}
  \begin{aligned}
    &\argmax_{w_x,w_y}&&\rho = E[z_xz_y]\\
    & \text{subject to}&& E[z_x^2] + \eta w_x^Hw_x\leq 1\\
    &&& E[z_y^2] + \eta w_y^Hw_y\leq 1.\\
  \end{aligned}
\end{equation}
We solve (\ref{eq:rcca_opt}) by taking the SVD of $\Creg = \left(\Rxxhat+\eta
  I_{p}\right)^{-1/2}\Rxyhat\left(\Ryyhat +\eta I_{q}\right)^{-1/2}$. Let
  $FKG^H$ be the SVD of $\Creg$ where $F=[f_1,\dots,f_{p}]$,
  $K=\diag(k_1,\dots,k_{\min\left(p,q\right)})$, and $G=[g_1,\dots,g_{q}]$. The
  solution to RCCA is
\beq\label{eq:rcca_sol}\ba
    & \rho =k_1 \\
    & w_x=(\Rxxhat+\eta I_{p})^{-1/2}f_1 \\
    & w_y = (\Ryyhat+\eta I_{q})^{-1/2}g_1.\\
\ea\eeq
Higher order canonical vector and correlation pairs are again computed using successive singular
value and vector pairs of $\Creg$.

\subsection{ICCA}

We repeat the informative CCA (ICCA) algorithm presented in Chapter
\ref{sec:chpt_cca_det}, first proposed by Nadakuditi
\cite{nadakuditi2011fundamental}. ICCA can avoid the performance loss in the sample
deficient regime by first trimming the individual data matrices to only include
informative subspace components. Let $\widetilde{X}=F_xK_xG_x^T$ and
$\widetilde{Y}=F_yK_yG_y^T$ be the data SVDs for our data matrices. Define the trimmed data
matrices
\be\ba
& \widetilde{F}_x = F_x(:,1:r_x) && \widetilde{G}_x = G_x(:,1:r_x)\\
& \widetilde{F}_y = F_y(:,1:r_y) && \widetilde{G}_y = G_y(:,1:r_y)\\
\ea\ee 
where $r_x$ and $r_y$ are the number of informative components in the first
and second datasets, respectively. To determine the number of informative components one
may employ techniques in \cite{asendorf2013performance, nadakuditi2008sample}. Using these trimmed data matrices, we form the matrix
used for ICCA,
\beq\label{eq:icca_chat}
  \widetilde{C} = \widetilde{F}_x\widetilde{G}_x^T\widetilde{G}_y\widetilde{F}_y^T,
\eeq
with SVD $\widetilde{C} = \widetilde{F}\widetilde{K}\widetilde{G}^H$, where $\widetilde{F}=[\widetilde{f}_1,\dots,\widetilde{f}_{r_x}]$, $\widetilde{K}=\diag(\widetilde{k}_1,\dots,\widetilde{k}_{\min\left(r_x,r_y\right)})$, and $\widetilde{G}=[\widetilde{g}_1,\dots,\widetilde{g}_{r_y}]$. ICCA returns the following informative correlation estimate and canonical vectors
\beq\label{eq:icca_rho}
\ba
&\rho = \widetilde{k}_1\\
&w_x = \Rxxhat^{-1/2}\widetilde{f}_1\\
&w_y= \Ryyhat^{-1/2}\widetilde{g}_1\\
\ea\eeq
Higher order canonical vector and correlation pairs are computed using successive singular
value and vector pairs of $\widetilde{C}$.