In this appendix we first derive the almost sure limit of the top empirical CCA
correlation estimates using the low-rank signal-plus-noise data model presented in Chapter
4. This is a recent result by \cite{bao2014canonical} but we present a similar derivation using
our data model and our random matrix theory notation. We then consider the accuracy of the
corresponding canonical vectors returned by empirical CCA. We derive a non-closed form
expression for this vector accuracy and discuss the approximations we make to compute the
needed terms. Finding a closed form expression for the canonical vector accuracy remains
future work.

\hiddensection{Model}

We repeat the data model from Chapter 4 for simplicity. Let $\xii\in\complex^{p\times 1}$
and $\yii\in\complex^{q\times 1}$ be modeled as
\beq\ba\label{eq:appen3:data_model}
&\xii = \Ux\sx + \zx\\
&\yii = \Uy\sy + \zy,\\
\ea\eeq
where $\Ux^H\Ux=I_{\kx}$, $\Uy^H\Uy=I_{\ky}$, $\zx\simiid\mathcal{CN}(0,I_p)$ and
$\zy\simiid\mathcal{CN}(0,I_q)$. Furthermore, assume that
\be\ba
&\sx\sim\mathcal{CN}(0,\Tx)\\
&\sy\sim\mathcal{CN}(0,\Ty),\\
\ea\ee
where $\Tx=\diag\left(\left(\tx_1\right)^2,\dots,\left(\tx_{\kx}\right)^2\right)$ and
$\Ty=\diag\left(\left(\ty_{1}\right)^2,\dots,\left(\ty_{\ky}\right)^2\right)$. Assume that $\zx$ and $\zy$ are
mutually independent and independent from both $\sx$ and $\sy$. Finally, assume that
\be
\E{\sx \sy^H} \defeq \Kxy = \Tx^{1/2}\Pxy\Ty^{1/2}
\ee
where the entries of $\Pxy$ are $-1\leq \rho_{kj} \leq 1$ and represent the correlation
between $\sx^{(k)}$ and $\sy^{(j)}$. For reasons to be made clear later, define 
\be
\Kxytil = \left(\Tx+I_{\kx}\right)^{-1/2}\Kxy\left(\Ty+I_{\ky}\right)^{-1/2}
\ee
and define the singular values of $\Kxytil$ as
$\kappa_1,\dots,\kappa_{\min(\kx,\ky)}$. Under this model, we define the following 
covariance matrices  
\beq\label{eq:true_scm}\ba
&\E{\xii\xii^H} = \Ux\Tx\Ux^H + I_p \defeq \Rxx\\
&\E{\yii\yii^H} = \Uy\Ty\Uy^H + I_q \defeq \Ryy\\
&\E{\xii\yii^H} = \Ux\Kxy\Uy^H \defeq \Rxy.\\
\ea\eeq
We define the rank of $\Rxy$ to be $k$.


\hiddensection{Almost Sure Limit of CCA Eigenvalues}

Bao et. al \cite{bao2014canonical} solve for the canonical correlation estimates in the following
setting
\beq\label{eq:bao}
\left[\begin{array}{c}X \\ Y\end{array}\right] = \left[\begin{array}{cc}I_p & R \\ R^H & I_q\end{array}\right]^{1/2}\left[\begin{array}{c}W_1 \\ W_2\end{array}\right]
\eeq
where $W_1$ has independent columns that are $\mathcal{N}(0,I_p)$ and $W_2$ has
independent columns that are $\mathcal{N}(0,I_q)$; $W_1$ is independent of $W_2$ and $R =
\diag(\sqrt{r_1},\dots,\sqrt{r_k},0,\dots 0)$. In this setup, $[X^H\,\,\, Y^H]^H$ has covariance
matrix 
\be
\left[\begin{array}{cc}I_p & R \\ R^H & I_q\end{array}\right]
\ee
and we may view (\ref{eq:bao}) as a sample from this covariance matrix. Here, we first show
that the above data model in (\ref{eq:appen3:data_model}) can be transformed via
invertible transformations to achieve the 
form of (\ref{eq:bao}). Using our own notation, we then provide an alternative but
equivalent derivation to Bao et al. for the almost sure limit of the correlation estimates of
empirical CCA. In this setting, we assume that $n>p+q$ as below this limit, we know via simple
geometric arguments that $\rhohatcca=1$ deterministically.

We now show that our data model in (\ref{eq:appen3:data_model}) may be formulated as (\ref{eq:bao}). In
our model, 
\be
\left[\begin{array}{c}X \\ Y\end{array}\right] = \left[\begin{array}{cc}R_{xx} & R_{xy} \\ R_{yx} & R_{yy}\end{array}\right]^{1/2}\left[\begin{array}{c}W_1 \\ W_2\end{array}\right]
\ee
where $W_1$ and $W_2$ are the same as above. This views the data matrices as a sample from a
covariance matrix. 
Define
\be
\left[\begin{array}{c}\widetilde{X} \\ \widetilde{Y}\end{array}\right] =
\left[\begin{array}{cc}\Rxx & 0 \\ 0 &
    \Ryy\end{array}\right]^{-1/2}\left[\begin{array}{c}X \\ Y\end{array}\right]. 
\ee
In this setup, $[X^H\,\,\, Y^H]^H$ has covariance matrix 
\be
\left[\begin{array}{cc}I_p & M \\ M^H & I_q\end{array}\right]
\ee
where 
\be
M = \Rxx^{-1/2}\Rxy\Ryy^{-1/2}.
\ee
With the definitions of the population covariance matrices for our data model,
\be
M = \Ux\left(\Tx+I_{\kx}\right)^{1/2}\Tx^{1/2}\Pxy\Ty^{1/2}\left(\Ty+I_{\ky}\right)^{-1/2}\Uy^H.
\ee
Define $F_MD_MG_M^H$ to be the SVD of M, noting that $D_M$ has at most $k$ nonzero
singular values. Then make the transformation
\be\ba
&\left[\begin{array}{c}\widetilde{\widetilde{X}} \\ \widetilde{\widetilde{Y}}\end{array}\right] &&=
\left[\begin{array}{cc}F_M^H & 0 \\ 0 &
    G_M^H\end{array}\right]^{1/2}\left[\begin{array}{c}\widetilde{X} \\
    \widetilde{Y}\end{array}\right] =  \left[\begin{array}{cc}F_M^H\Rxx^{-1/2} & 0 \\ 0 &
    G_M^H\Ryy^{-1/2}\end{array}\right]^{1/2}\left[\begin{array}{c} X \\
    Y \end{array}\right]\\
&&& = \left[\begin{array}{cc}I_p & D_M \\ D_M^H &
    I_q\end{array}\right]^{1/2}\left[\begin{array}{c}W_1 \\ W_2\end{array}\right].
\ea\ee
Therefore, from this transformation, we are in the same setting as Bao et al. with, for
$i=1,\dots,k$,
\be
d_i = \sqrt{r_i}.
\ee
We note that $d_i$ are the singular values of $\Kxytil$, which we defined as
$\kappa_1,\dots,\kappa_k$. If we are in the special case where $\Pxy=\diag(\rho_1,\dots,\rho_{k})$, then 
\be
d_i =
 \frac{\tx_i\ty_i\rho_i}{\sqrt{\left(\tx_i\right)^2+1}\sqrt{\left(\ty_i\right)^2+1}}.
\ee

Next we proceed with an analogous proof of Bao et al. As noted in their paper, transforming $X$
and $Y$ to $\widetilde{\widetilde{X}}$ and $\widetilde{\widetilde{Y}}$ preserves the
canonical correlation estimates because the transformation matrix is
non-singular. Our target matrix in CCA is
\beq\label{eq:cca_target}
\Ccca = \Rxx^{-1}\Rxy\Ryy^{-1}\Rxy^H.
\eeq
Clearly, making invertible transformations of $X$ and $Y$ preserves the eigenvalues of
(\ref{eq:cca_target}). We proceed assuming $X$ and $Y$ are the transformed versions
$\widetilde{\widetilde{X}}$ and $\widetilde{\widetilde{Y}}$ to ease notation.

First for $i=1,\dots,k$, define
\be
\alpha_i = \frac{\sqrt{1+d_i}+\sqrt{1-d_i}}{2},\,\,\,\beta_i=\frac{\sqrt{1+d_i}-\sqrt{1-d_i}}{2}
\ee
and the matrices 
\be
P_1 = \left[\begin{array}{cc}\diag(\alpha_1,\dots,\alpha_k) & 0 \\ 0 &
    I_{p-k}\end{array}\right], P_2 = \left[\begin{array}{cc}\diag(\alpha_1,\dots,\alpha_k)
    & 0 \\ 0 & I_{q-k}\end{array}\right],\,\,
\ee
\be
P_3 = \left[\begin{array}{cc}\diag(\beta_1,\dots,\beta_k)  & 0 \\ 0 & 0\end{array}\right].
\ee
With these definitions, we observe that
\be
\left[\begin{array}{cc}I_p & D_M \\ D_M^H &
    I_q\end{array}\right]^{1/2} = \left[\begin{array}{cc}P_1 & P_3 \\ P_3^H &
    P_2\end{array}\right].
\ee
Defining the transformation (again re-using notation for simplicity)
\be
\left[\begin{array}{c}X \\ Y\end{array}\right] =
\left[\begin{array}{cc}P_1^{-1} & 0 \\ 0 &
    P_2^{-1}\end{array}\right]\left[\begin{array}{c}X \\ Y\end{array}\right], 
\ee
we have that 
\be
\left[\begin{array}{c}X \\ Y\end{array}\right] =
\left[\begin{array}{cc}I_p & P \\ P &
    I_q\end{array}\right]\left[\begin{array}{c}W_1 \\ W_2\end{array}\right], 
\ee
where 
\be
P = \left[\begin{array}{cc}\diag(\tau_1,\dots,\tau_k) & 0 \\ 0 &
    0\end{array}\right],
\ee
with $\tau_i=\beta_i/\alpha_i$. Next, define
\be
Q = \left[\begin{array}{cc}\diag(2\tau_1/\left(1+\tau_1^2\right),\dots,2\tau_k/\left(1+\tau_k^2\right)) & 0 \\ 0 &
    0\end{array}\right],
\ee
and
\be
W = \left(I-QP^H\right)W_1 + (P-Q)W_2,
\ee
so that by construction $W$ and $Y$ are independent and
\be
X = W + QY.
\ee
The covariance matrices for $Y$ and $W$ are
\be\ba
&R_W = \E{\frac{1}{n}WW^H} = \left[\begin{array}{cc}\diag(1+\frac{\tau_1^4-3\tau_1^2}{1+\tau_1^2},\dots,1+\frac{\tau_k^4-3\tau_k^2}{1+\tau_k^2}) & 0 \\ 0 &
    I_{p-k}\end{array}\right]\\
&R_Y = \E{\frac{1}{n}YY^H} = \left[\begin{array}{cc}\diag(1+\tau_1,\dots,1+\tau_k) & 0 \\ 0 &
    I_{q-k}\end{array}\right].
\ea\ee
Finally making the transformation
\be
\left[\begin{array}{c}\widetilde{X} \\ \widetilde{Y}\end{array}\right] =
\left[\begin{array}{cc}R_W^{-1/2} & 0 \\ 0 &
    R_Y^{-1/2}\end{array}\right]\left[\begin{array}{c}X \\ Y\end{array}\right], 
\ee
yields
\begin{enumerate}
\item $\widetilde{Y}$ has independent columns that are $\mathcal{N}(0,I_q)$
\item $\widetilde{W}=R_W^{-1/2}W$ has independent columns that are $\mathcal{N}(0,I_p)$.
\item $\widetilde{X}$ has independent columns that are
  $\mathcal{N}(0,R_W^{-1/2}QR_YQ^HR_W^{-1/2})$
\item $\widetilde{W}$ and $\widetilde{Y}$ are independent.
\end{enumerate}
Denoting
\be
T = R_W^{-1/2}QR_Y^{1/2} = \left[\begin{array}{cc}\diag(t_1,\dots,t_k) & 0 \\ 0 &
    0\end{array}\right],
\ee
with $t_i = 2\tau_i/(1-\tau_i^2)$, we finally arrive at the setting (again dropping the tildes)
\beq\label{eq:cca_reduced}
\left[\begin{array}{c}X \\ Y \end{array}\right] = \left[\begin{array}{c}W+TY \\
    Y \end{array}\right]. 
\eeq
This is wonderful because we now have a perturbation model for $X$ and $Y$. Defining the
sample covariance matrices of our matrices as
\be\ba
&\Sxx = \frac{1}{n}XX^H\\
&\Syy = \frac{1}{n}YY^H\\
&\Sxy = \frac{1}{n}XY^H\\
&\Syx = \frac{1}{n}YX^H\\
&\Sww = \frac{1}{n}WW^H\\
&\Swy = \frac{1}{n}WY^H\\
&\Syw = \frac{1}{n}YW^H,\\
\ea\ee
we have the relationship. that
\be\ba
& \Sxx = \Sww + T\Syw + \Swy T^H + T^H\Syy T^H\\
& \Sxy = \Swy + T\Syy\\
& \Syx = \Syw + \Syy T^H.\\
\ea\ee
Therefore the CCA matrix in (\ref{eq:cca_target}) becomes a low rank matrix plus a product
of independent noise matrices. We show this beginning with the characteristic equation for
(\ref{eq:cca_target}). 
\be\ba
&0 &&= \det\left(\Sxx^{-1}\Sxy\Syy^{-1}\Syx - \lambda I\right)\\
&&& = \det\left(\Sxy\Syy^{-1}\Syx - \lambda \Sxx\right)\\
&&& = \det\left(\left(\Swy + T\Syy\right)\Syy^{-1}\left(\Syw + \Syy T^H\right) - \lambda
  \left(\Sww + T\Syw + \Swy T^H + T^H\Syy T^H\right)\right)\\
&&& = \det\left(\Swy\Syy^{-1}\Syw - \lambda\Sww + (1-\lambda)\underbrace{\left(T\Syw+\Swy T^H + T\Syy T^H\right)}_{\Delta}\right)\\
\ea\ee
This give a nice low rank perturbation of a random matrix product. Therefore, if $\lambda$
is not an eigenvalue of $\Sww^{-1}\Swy\Syy^{-1}\Syw$. we have
\be
0=\det\left(I_p + (1-\lambda)\left(\Swy\Syy^{-1}\Syw - \lambda\Sww\right)^{-1}\Delta\right).
\ee
Examining $T$, we see that the rank of $T$ is at most $k$ as therefore $\Delta$ will be
low rank. The structure of $\Delta$ is ugly but necessary for the remainder of the proof
and we slightly alter the notation of Bao et al. We have that
\be
\Delta = UV^H
\ee
where
\be\ba
&U = \left[A_1,\dots,A_k,F_1,\dots,F_k\right]\\
&V = \left[B_1,\dots,B_k,C_1,\dots,C_k\right]\\
\ea\ee
where
\be\ba
& A_i = \left[\chi_{ii}e_i, t_ie_i,t_iu_i\right]\\
& B_i = \left[e_i,u_i,e_i\right]\\
& C_i = \left[\underbrace{e_i,\dots,e_i}_{k-1}\right]\\
& F_1 = \left[\chi_{12}e_1,\dots,\chi_{1k}e_k\right]\\
& F_i = \left[\chi_{i1}e1,\dots\chi_{i,i-1}e_{i-1},\chi_{i,i+1}e_{i+1},\dots,\chi_{ik}e_k\right]\\
\ea\ee
where $e_i$ is $i$-th elementary vector, $u_i$ is the $i$-th column of $\Swy$ and
$\chi_{ij}=t_it_j\Syy(i,j)$. With these definitions, and using the identity that
$\det(I+AB)=\det(I+BA)$, we have that 
\be
0 = \det(I_{k^2+2k} + (1-\lambda)V^H\left(\Swy\Syy^{-1}\Syw - \lambda\Sww\right)^{-1}U).
\ee
In this form, we have our standard characteristic equation of a low rank perturbation of a
random matrix. In this case $\Delta$ is our perturbation and $\Sww^{-1}\Swy\Syy^{-1}\Syw$
is the random matrix. This highlights the importance of the previous transformations
needed to write $X=W+TY$. The random matrix product is now of independent components and
we can therefore compute statistics on its eigenvalues.

First define
\beq\label{eq:M}
M(\lambda)= I_{k^2+2k} + (1-\lambda)V^H\left(\Swy\Syy^{-1}\Syw - \lambda\Sww\right)^{-1}U
\eeq
and based on the structure of $U$ and $V$, we have that 
\be
M(\lambda) = I_{k^2+2k} + (1-\lambda)\blkdiag(G_1(\lambda),\dots,G_k(\lambda),0,\dots,0),
\ee
where 
\be
G_i(\lambda) = \left[\begin{array}{ccc}t_i^2f(\lambda) & t_if(\lambda) & 0 \\
  0 & 0 & t_i h(\lambda)\\ t_i^2 f(\lambda) &t_i f(\lambda) & 0\end{array}\right]
\ee
with
\be\ba
& f(\lambda) = e_i^H\left(\Swy\Syy^{-1}\Syw - \lambda\Sww\right)^{-1}e_i^H\\
& h(\lambda) = u_i^H\left(\Swy\Syy^{-1}\Syw - \lambda\Sww\right)^{-1}u_i^H.\\ 
\ea\ee
We get such a nice structure for $M(\lambda)$ in (\ref{eq:M}) due to many cancellations of
terms such as
\be\ba
& e_i^H\left(\Swy\Syy^{-1}\Syw - \lambda\Sww\right)^{-1}e_j^H\\
& u_i^H\left(\Swy\Syy^{-1}\Syw - \lambda\Sww\right)^{-1}u_j^H\\ 
& e_i^H\left(\Swy\Syy^{-1}\Syw - \lambda\Sww\right)^{-1}u_j^H.\\ 
\ea\ee
Therefore, to solve our characteristic equation, we may look at the sub-blocks of $M$ of
the form $I_3+(1-\lambda)G_i(\lambda)$. The determinant of this $3\times 3$ matrix is
\beq\label{eq:det}
1 + (1-\lambda)t_i^2f(\lambda) - (1-\lambda)^2t_i^2f(\lambda)h(\lambda).
\eeq
To complete the proof, we must find closed form expressions for $f(\lambda)$ and $h(\lambda)$ and
substitute them into (\ref{eq:det}) to solve for $\lambda$. 

First, define the projection matrix $P_Y= Y^H\left(YY^H\right)^{-1}Y$ and the matrices $E
= \frac{1}{n}WP_YW^H$ and $H=\frac{1}{n}W(I-P_Y)W^H$. Therefore, with these definitions
\be
\left(\Swy\Syy^{-1}\Syw - \lambda\Sww\right)^{-1} = \left(E - \lambda(E+H)\right)^{-1} =
\left((1-\lambda)E - \lambda H\right)^{-1}.
\ee
These definitions are wonderful because $E$ and $H$ are independent by construction and 
\be\ba
& E\sim \text{Wishart}_p(I_p,q)\\
& H\sim \text{Wishart}_p(I_p,n-q).\\
\ea\ee
This formulation is sufficient for $f(\lambda)$ but not $h(\lambda)$. Examining
$h(\lambda)$, we have that
\be
h(\lambda) = u_1^H\left(\Swy\Syy^{-1}\Syw - \lambda\Sww\right)^{-1}u_1 =
e_1^H\Syw\left(\Swy\Syy^{-1}\Syw - \lambda\Sww\right)^{-1}\Swy e_1.
\ee
Next, we attempt to get a similar expression for this additional matrix product. Defining 
\be
\Phi(\lambda)=\Syw\left(\Swy\Syy^{-1}\Syw - \lambda\Sww\right)^{-1}\Swy 
\ee
and applying the Woodbury matrix identity, we have
\be\ba
& \Phi(\lambda) && =
\Syw\Sww^{-1/2}\left(\Sww^{-1/2}\Swy\Syy^{-1}\Syw\Sww^{-1/2} -
  \lambda I_p\right)^{-1}\Sww^{-1/2}\Swy \\
&&& = \Syw\Sww^{-1/2}\left[-\frac{1}{\lambda} -
  \frac{1}{\lambda^2}\Sww^{-1/2}\Swy\left(\Syy -
    \frac{1}{\lambda}\Syw\Sww^{-1}\Swy\right)^{-1}\Swy\Sww^{-1/2}\right]\Sww^{-1/2}\Swy .
\ea\ee
Next define $A=\Syw\Sww^{-1}\Swy=\frac{1}{n}YP_wY^H$ and $B=\frac{1}{n}Y(I-P_w)Y^H$,
similar to above. Recall that with these definitions,
\be\ba
& A\sim \text{Wishart}_q(I_q,p)\\
& B\sim \text{Wishart}_q(I_q,n-p).\\
\ea\ee
With these definitions, we have
\be\ba
& \Syw\left(\Swy\Syy^{-1}\Syw - \lambda\Sww\right)^{-1}\Swy && = -\frac{1}{\lambda}A +
\frac{1}{\lambda}A\left(A-\lambda(A+B)\right)^{-1}A\\ 
&&& = -\frac{1}{\lambda}A +
\frac{1}{\lambda}A\left((1-\lambda)A-\lambda B\right)^{-1}A\\
\ea\ee
After another application of the Woodbury matrix identity, we have that
\be\ba
& \Phi(\lambda)
&& = -\frac{1}{\lambda}A +
\frac{1}{\lambda}A\left[\frac{1}{1-\lambda}A^{-1}-\frac{1}{(1-\lambda)^2}A^{-1}\left(-\frac{1}{\lambda}B^{-1}
  + \frac{1}{1-\lambda}A^{-1}\right)^{-1}A^{-1}\right]A\\
&&& = \frac{1}{1-\lambda}A - \frac{1}{\lambda(1-\lambda)^2}\left(-\frac{1}{\lambda}B^{-1}
  +\frac{1}{1-\lambda}A^{-1}\right)^{-1}\\
&&& = \frac{1}{1-\lambda}A + \frac{1}{1-\lambda}\left((1-\lambda)B^{-1} -\lambda
  A^{-1}\right)^{-1}.\\ 
\ea\ee
While this may look ugly, it is quite useful.

Recall that the Stieltjes transform of a spectral distribution of a matrix X with
eigenvalues $\gamma_1,\dots,\gamma_n$ is
\beq\label{eq:stiel}\ba
&m_{\mu_X}(z) &&= \int\frac{1}{\gamma-z}d_{\mu_X}\\
&&& =\frac{1}{n}\Tr(\left(X-z I\right)^{-1})\,\,\,\text{in the finite case}\\
&&&=\frac{1}{n}\sum_{i=1}^n\frac{1}{\gamma_i-z}
\ea\eeq
In addition, recall the R-transform of a spectral distribution is 
\beq\label{eq:r_trans}
R_{\mu_X}(z) = K_{\mu_X}(z) - \frac{1}{z},
\eeq
where $K_{\mu_X}$ is the Blue function of the spectral distribution with the property
\beq\label{eq:blue}
-m_{\mu_X}(K_{\mu_X}(z)) = K_{\mu_X}(-m_{\mu_X}(z)) = z.
\eeq
Specifically, we have the relationship that
\be\ba
&R_{\mu_X}(-m_{\mu_X}(z)) &&= K_{\mu_X}(-m_{\mu_X}(z)) +\frac{1}{m_{\mu_X}(z)}\\
&&& = z + \frac{1}{m_{\mu_X}(z)}.\\
\ea\ee
Therefore we can recover the Stieltjes transform from the R-transform. A very nice property
of the R-transform is free additive convolution. Mainly, if we have matrices $X_1$ and $X_2$, 
\beq\label{eq:r_plus}
R_{X_1+X_2}(z) = R_{X_1}(z) + R_{X_2}(z).
\eeq

With these definitions in mind, we first note that $m_{\mu_X}(0) =
\frac{1}{n}\Tr(X^{-1})$. This is extremely helpful for our problem! Define the matrices
\be\ba
& J = (1-\lambda)E + \lambda H\\
& \widetilde{J} = (1-\lambda)B^{-1} + \lambda A^{-1}.\\
\ea\ee
To solve for $f(\lambda)$ and $h(\lambda)$, we need so solve for the Stieltjes transforms
of $J$ and $\widetilde{J}$. To accomplish this, we first will solve for the Stieltjes
transforms of the component matrices, find the associated R-transform, use
(\ref{eq:r_plus}) to find the R-transform of the sum, and then transform back to Stieltjes
transforms. Recall that $E,H,A,B$ are all Wishart random matrices. The basic Stieltjes
transform for a Wishart random matrix $X$ with parameter $c$ is
\be
m_{\mu_X}(z) = \frac{(1-c) - z + \sqrt{\left(z-1-c\right)^2-4c}}{2cz}.
\ee
We define $\widetilde{E} = (1-\lambda)E$, $\widetilde{H} = -\lambda H$, $\widetilde{B} =
(1-\lambda)B^{-1}$, and $\widetilde{A}=-\lambda A^{-1}$. The Stieltjes transforms of the
limiting spectral densities of these matrices are (through change of variables and some
calculation)
\be\small\ba
&m_{\mu_{\widetilde{E}}}(z) = \frac{(1-\lambda)(c_y-c_x) - z +
  \sqrt{(z-(1-\lambda)(c_x+c_y))^2 - 4(1-\lambda)^2c_xc_y}}{2(1-\lambda)c_x z}\\
&m_{\mu_{\widetilde{H}}}(z) = \frac{\lambda(1-c_y-c_x) + z -
  \sqrt{(z+\lambda(1+c_x-c_y))^2 - 4c_x(\lambda)^2(1-c_y)}}{2\lambda c_x z}\\
&m_{\mu_{\widetilde{A}}}(z) = -z^{-1} - \frac{c_x-c_y +\lambda z^{-1} -
  \sqrt{(z^{-1}\lambda +c_x+c_y)^2 - 4c_xc_y}}{2c_y z}\\
&m_{\mu_{\widetilde{B}}}(z) = -z^{-1} - \frac{1-c_x-c_y -(1-\lambda) z^{-1} +
  \sqrt{(z^{-1}(1-\lambda) -(1-c_x+c_y))^2 - 4(1-c_x)c_y}}{2c_y z}\\
\ea\ee
Using (\ref{eq:r_trans}) and (\ref{eq:blue}), we have that the R-transforms of these
expressions are
\be\ba
& R_{\mu_{\widetilde{E}}}(w) = \frac{(1-\lambda)c_y}{1-(1-\lambda)c_xw}\\
& R_{\mu_{\widetilde{H}}}(w) = -\frac{\lambda (1-c_y)}{1+\lambda c_xw}\\
& R_{\mu_{\widetilde{A}}}(w) = \frac{c_x-c_y-\sqrt{(c_x-c_y)^2+4\lambda c_y w}}{2c_yw}\\
& R_{\mu_{\widetilde{B}}}(w) = \frac{1-c_x-c_y - \sqrt{(1-c_x-c_y)^2+4(\lambda-1)c_y w}}{2c_yw}\\
\ea\ee

By observation we have that
\be
f(\lambda) \to m_{\mu_J}(0).
\ee
We know by definition of $J$ and using (\ref{eq:r_plus}), (\ref{eq:r_trans}), and
(\ref{eq:blue}) that 
\be\ba
&R_{\mu_J}(w) &&= R_{\mu_{\widetilde{E}}}(w) + R_{\mu_{\widetilde{H}}}(w)\\
&R_{\mu_J}(-m_{\mu_J}(0)) &&= R_{\mu_{\widetilde{E}}}(-m_{\mu_J}(0)) +
R_{\mu_{\widetilde{H}}}(-m_{\mu_J}(0))\\ 
& K_{\mu_{J}}(-m_{\mu_J}(0)) + \frac{1}{m_{\mu_J}(0)} && = \frac{(1-\lambda)c_y}{1+(1-\lambda)c_x m_{\mu_J}(0)}-\frac{\lambda
  (1-c_y)}{1-\lambda c_x m_{\mu_J}(0)}\\
& \frac{1}{m_{\mu_J}(0)} && = \frac{(1-\lambda)c_y}{1+(1-\lambda)c_x m_{\mu_J}(0)}-\frac{\lambda
  (1-c_y)}{1-\lambda c_x m_{\mu_J}(0)} \\
\ea\ee
Solving the above for $m_{\mu_J}(0)$ yields an expression for $f(\lambda)$
\beq\label{eq:f_lam}
f(\lambda) = \frac{-(c_y-c_x+2\lambda c_x - \lambda) - \sqrt{\lambda^2 + (4c_xc_y
    -2c_x-2c_y)\lambda + (c_x-c_y)^2}}{2\lambda(1-\lambda)(c_x^2-c_x)}.
\eeq

We proceed similarly to get an expression for $h(\lambda)$. However, we note that
\be\ba
&h(\lambda) &&= \frac{1}{1-\lambda}e_i^HAe_i +
\frac{1}{1-\lambda}e_1\left((1-\lambda)B^{-1} + \lambda A^{-1}\right)^{-2}e_1\\
&&& \to c_x+\frac{1}{1-\lambda}m_{\mu_{\widetilde{J}}}(0)\\
\ea\ee
as $e_i^HAe_i$ converges to the expected value of the limiting spectral density of $A$. As
$A$ is Wishart with parameter $c_x$, we know that this expectation is simply
$c_x$. Therefore, we are left to solve for $m_{\mu_{\widetilde{J}}}(0)$. We solve this
again via the R-transform
\be\ba
&R_{\mu_{\widetilde{J}}}(w) &&= R_{\mu_{\widetilde{A}}}(w) + R_{\mu_{\widetilde{B}}}(w)\\
&R_{\mu_{\widetilde{J}}}(-m_{\mu_{\widetilde{J}}}(0)) &&=
  R_{\mu_{\widetilde{A}}}(-m_{\mu_{\widetilde{J}}}(0))
    +R_{\mu_{\widetilde{B}}}(-m_{\mu_{\widetilde{J}}}(0))\\  
& K_{\mu_{\widetilde{J}}}(-m_{\mu_{\widetilde{J}}}(0)) +
\frac{1}{m_{\mu_{\widetilde{J}}}(0)} && =  -\frac{c_x-c_y-\sqrt{(c_x-c_y)^2-4\lambda c_y
    m_{\mu_{\widetilde{J}}}(0)}}{2c_y m_{\mu_{\widetilde{J}}}(0)}\\
&&& - \frac{1-c_x-c_y -
  \sqrt{(1-c_x-c_y)^2-4(\lambda-1)c_y m_{\mu_{\widetilde{J}}}(0)}}{2c_y m_{\mu_{\widetilde{J}}}(0)}\\
& \frac{1}{m_{\mu_{\widetilde{J}}}(0)} && =-\frac{c_x-c_y-\sqrt{(c_x-c_y)^2-4\lambda c_y
    m_{\mu_{\widetilde{J}}}(0)}}{2c_y m_{\mu_{\widetilde{J}}}(0)}\\
&&& - \frac{1-c_x-c_y -
  \sqrt{(1-c_x-c_y)^2-4(\lambda-1)c_y m_{\mu_{\widetilde{J}}}(0)}}{2c_y m_{\mu_{\widetilde{J}}}(0)}.\\\\
\ea\ee
Solving the above for $m_{\mu_{\widetilde{J}}}(0)$ yields
\be
m_{\mu_{\widetilde{J}}}(0) = \frac{c_x+c_y-2c_xc_y -\lambda + \sqrt{\lambda^2 + (4c_xc_y -
    2c_x-2c_y)\lambda + (c_x-c_y)^2}}{2c_y}
\ee 
Therefore, we have
\beq\label{eq:h_lam}
h(\lambda) = \frac{c_x}{1-\lambda} + \frac{c_x+c_y-2c_xc_y -\lambda + \sqrt{\lambda^2 + (4c_xc_y -
    2c_x-2c_y)\lambda + (c_x-c_y)^2}}{2c_y(1-\lambda)}.
\eeq
To conclude, we must substitute (\ref{eq:h_lam}) and (\ref{eq:f_lam}) into (\ref{eq:det})
and solve for $\lambda$. This is largely an algebra problem and we point the reader to Bao
et al.
if interested. After some calculation we arrive at the final result
\be
\rhohatcca^{(i)} \convas \begin{cases} \sqrt{d_i^2\left(1-c_x+\frac{c_x}{d_i^2}\right)\left(1-c_y+\frac{c_y}{d_i^2}\right)} & d_i^2 \geq r_c \\ \sqrt{d_r} &d_i^2<r_c\end{cases}
\ee
where
\be\ba
& r_c = \frac{c_xc_y+\sqrt{c_yc_y(1-c_x)(1-c_y)}}{(1-c_x)(1-c_y) + \sqrt{c_xc_y(1-c_x)(1-c_y)}}\\
& d_r = c_x+c_y-2c_xc_y+2\sqrt{c_xc_y(1-c_x)(1-c_y)}.
\ea\ee

\hiddensection{Canonical Vectors}

We now solve for the accuracy of the canonical vectors in empirical CCA. We consider,
without loss of generality, the accuracy of the canonical vector of only one dataset. Let
$X$ and $Y$ be drawn from (\ref{eq:appen3:data_model}). With the same definition of the
target matrix $\Ccca$, the estimated canonical vectors, $\wxhat^{(i)}$ solves the
generalized eigenvalue problem $C\wxhat^{(i)} = \rhohatcca^2\wxhat^{(i)}$. Recall from
Chapter 5 that the unit-norm
population canonical vector that we are trying to estimate is
\be
\wx^{(i)} = \frac{\Rxx^{-1/2}\Ux
  U_{\widetilde{K}}(:,i)}{\sqrt{U_{\widetilde{K}}(:,i)^H\Ux^H\Rxx^{-1}\Ux
    U_{\widetilde{K}}(:,i)}} 
\ee
where $U_{\widetilde{K}}$ are the left singular vectors of $\Kxytil$. In this section, we want to
find a closed form expression for $|\langle \wx^{(i)}, \wxhat^{(i)}\rangle|^2$. We first introduce the change of variables
similar to the correlation computation
\be\ba
& \widetilde{X} = F_M^H\Rxx^{-1/2}X\\
& \widetilde{Y} = G_M^H\Ryy^{-1/2}Y\\
\ea\ee
where $F_MD_MG_M^H$ is the SVD of $M$, defined above. Then with this transformation, we have
\be
\Ccca = \left(R_{xx}^{1/2}F_M\right)\underbrace{\left(\widetilde{X}\widetilde{X}^H\right)^{-1}\widetilde{X}\widetilde{Y}^H\left(\widetilde{Y}\widetilde{Y}^H\right)^{-1}\widetilde{Y}\widetilde{X}^H}_{\widetilde{C}}\left(F_M^HR_{xx}^{1/2}\right).
\ee
Then, if $\widetilde{u}^{(i)}$ is a unit-norm eigenvector of $\widetilde{C}$, via the similarity
transform, 
\be
\wxhat^{(i)} = \frac{\left(F_M^HR_{xx}^{1/2}\right)^{-1}\widetilde{u}^{(i)}}{
   \sqrt{\widetilde{u}^{(i)H}\left(F_M^HR_{xx}^{1/2}\right)^{-H}\left(F_M^HR_{xx}^{1/2}\right)^{-1}
     \widetilde{u}^{(i)}}}.  
\ee
 Therefore
\beq\label{eq:cca_ip}\ba
&|\langle \wx^{(i)}, \wxhat^{(i)}\rangle|^2 && =
\frac{\left(U_{\widetilde{K}}(:,i)^H\Ux^H\Rxx^{-1/2}\left(F_M^HR_{xx}^{1/2}\right)^{-1}\widetilde{u}^{(i)}\right)^2}{\left(U_{\widetilde{K}}(:,i)^H\Ux^H\Rxx^{-1}\Ux
    U_{\widetilde{K}}(:,i)\right)\left(\widetilde{u}^{(i)H}\left(F_M^HR_{xx}^{1/2}\right)^{-T}\left(F_M^HR_{xx}^{1/2}\right)^{-1}
     \widetilde{u}^{(i)}\right)}\\
&&& = \frac{\left(U_{\widetilde{K}}(:,i)^H\Ux^H\Rxx^{-1}F_M\widetilde{u}^{(i)}\right)^2}{\left(U_{\widetilde{K}}(:,i)^H\Ux^H\Rxx^{-1}\Ux
    U_{\widetilde{K}}(:,i)\right)\left(\widetilde{u}^{(i)H}F_M^HR_{xx}^{-1}F_M\widetilde{u}^{(i)}\right)}.\\ 
\ea\eeq
Now due to the structure of $\Kxytil$ and $M$, we have that
\be
F_M = \left[\Ux U_{\widetilde{K}}\,\,\, \left(\Ux U_{\widetilde{K}}\right)^\perp \right],
\ee
which allows us to rewrite
\be
 |\langle \wx^{(i)}, \wxhat^{(i)}\rangle|^2 = \frac{\left(e_i^HF_M^H\Rxx^{-1}F_M\widetilde{u}^{(i)}\right)^2}{\left(e_i^HF_M^H\Rxx^{-1}F_Me_i\right)\left(\widetilde{u}^{(i)H}F_M^HR_{xx}^{-1}F_M\widetilde{u}^{(i)}\right)}.
\ee
Also note that we can write our unit norm eigenvector as
\be
\widetilde{u}^{(i)} = \sum_{j=1}^p (\widetilde{u}^{(i)H}e_j)e_j.
\ee
We note that by the Theorem \ref{th:w_ip}, for $j=1,\dots,\kx, j\neq i$,
$(\widetilde{u}^{(i)H}e_j)\convas 0$. Examining the structure of the population covariance
matrix of our data model, the final $p-\kx$ eigenvalues of $\Rxx$ are 1 so
we have that the last term in the denominator above is
\be
\widetilde{u}^{(i)H}F_M^HR_{xx}^{-1}F_M\widetilde{u}^{(i)} \convas
\left(e_i^HF_M^H\Rxx^{-1}F_Me_i\right)\left(\widetilde{u}^{(i)H}e_i\right)^2 +
\left[1-\left(\widetilde{u}^{(i)H}e_i\right)^2\right].
\ee
By a similar argument, the term in the numerator is
\be
e_i^HF_M^H\Rxx^{-1}F_M\widetilde{u}^{(i)}\convas e_i^HF_M^H\Rxx^{-1}F_Me_i\left(\widetilde{u}^{(i)H}e_i\right).
\ee
Therefore, (\ref{eq:cca_ip}) becomes
\beq\label{eq:cca_ip_red}
|\langle \wx^{(i)}, \wxhat^{(i)}\rangle|^2 \convas \frac{\gamma_i
  \left(\widetilde{u}^{(i)H}e_i\right)^2}{(\gamma_i-1)\left(\widetilde{u}^{(i)H}e_i\right)^2+1},
\eeq
where
\be
\gamma_i = e_i^HF_M^H\Rxx^{-1}F_Me_i.
\ee
Thus, it suffices to find the accuracy of $\widetilde{u}^{(i)}$ with respect to $e_i$ to
solve (\ref{eq:cca_ip_red}).

We proceed by first noting that in the above CCA correlation derivation, we first
transformed $\widetilde{X}$ and $\widetilde{Y}$ by a series of invertible linear
transformations to get the final matrix perturbation form. While this does not affect the
eigenvalues of the target matrix, it does affect the eigenvectors and so we correct for
that here via similarity transformations. These transformations may be done in one step
and since the transformation matrices are all diagonal, it makes the story a little
easier. Specifically we have
\be\ba
& \widetilde{\widetilde{X}} = M_x\widetilde{X}\\
& \widetilde{\widetilde{Y}} = M_y\widetilde{Y}\\
\ea\ee
where
\be
M_x = \left[\begin{array}{cc}\diag(m_{x1},\dots,m_{x\kx}) & 0 \\ 0 & I_{p-\kx}\end{array}\right], M_y = \left[\begin{array}{cc}\diag(m_{y1},\dots,m_{y\ky}) & 0 \\ 0 & I_{q-\ky}\end{array}\right]
\ee
with
\be
m_{xi} = \frac{1}{\alpha_i\sqrt{1+\sqrt{\frac{\tau_i^4-3\tau_i^2}{1+\tau_i^2}}}},\,\,\, m_{yi} =
\frac{1}{\alpha_i\sqrt{1+\tau_i^2}},
\ee
where these parameter were defined in the correlation derivation. After these
transformations, our target matrix is 
\be
\widetilde{C} = M_x^{-1}\widetilde{\widetilde{C}}M_x. 
\ee
Again via a similarity transform, we
see that if $\widetilde{\widetilde{u}}$ is an eigenvector of $\widetilde{\widetilde{C}}$,
then $\widetilde{u} =
\frac{M_x\widetilde{\widetilde{u}}}{\sqrt{\widetilde{\widetilde{u}}^HM_x^2\widetilde{\widetilde{u}}}}$. Then
via a similar computation, we have
\be
\left|\langle e_i, \widetilde{u}^{(i)}\rangle\right|^2 =
\frac{m_{xi}^2\left(e_i^H\widetilde{\widetilde{u}}^{(i)}\right)^2}{\left(m_{xi}^2-1\right)\left(e_i^H\widetilde{\widetilde{u}}^{(i)}\right)^2+1}.
\ee
Therefore, solving for the eigenvector accuracy of $\widetilde{\widetilde{u}}$ we can
recover the accuracy of our canonical vector via
\beq\label{eq:cca_wx_acc}
\left|\langle \wx^{(i)}, \wxhat^{(i)}\rangle\right|^2=\frac{\gamma_im_{xi}^2\left|\langle e_i, \ttu^{(i)}\rangle\right|^2}{\left|\langle e_i, \ttu^{(i)}\rangle\right|^2\left(m_{xi}^2\left(\gamma_i-1\right)+m_{xi}^2-2\right)+1}.
\eeq
As $\gamma_i$ and $m_{xi}$ are parameters of our problems, what is left is to determine
the eigenvector accuracy of $\widetilde{\widetilde{u}}^{(i)}$. We proceed using the same
matrix perturbation model as in the eigenvalue derivation. Our master equation is
\be\ba
&\widetilde{\widetilde{C}}\ttu &&= \lambda\ttu\\
&S_{xy}S_{yy}^{-1}S_{yx}\ttu &&= \lambda S_{xx}\ttu\\
\ea\ee
We made these specific transformations to achieve the setting of
(\ref{eq:cca_reduced}). Using similar derivations as the eigenvalue setting, we arrive at
the low-rank matrix we desire, which just so happens to have a component of $e_i$.
\be\ba
&\left(S_{wy}S_{yy}^{-1}S_{yw} -\lambda S_{ww} + (1-\lambda)UV^H\right)\ttu^{(i)} && = 0
\ea\ee
The low rank matrix $UV^H$ will contain terms of the form $e_iu_i^H$, $u_ie_i^H$, and
$e_ie_i^H$. As shown in Theorem 2.7 c) in \cite{benaych2012singular}, the energy of $\ttu^{(i)}$ lying in orthogonal
components of $e_{j\neq i}$ will be zero and hence for notational simplicity, we ignore
them going forward. Proceeding, we have
\be\ba
&\left(S_{wy}S_{yy}^{-1}S_{yw} -\lambda S_{ww} + (1-\lambda)t_i(e_iu_i^H + u_ie_i^H) +
  (1-\lambda)t_i^2S_{yy}(i,i)e_ie_i^H\right)\ttu^{(i)} &&= 0\\
\ea\ee
\be\ba
&\left(S_{wy}S_{yy}^{-1}S_{yw} -\lambda S_{ww} + (1-\lambda)t_i(e_iu_i^H + u_ie_i^H) \right)\ttu^{(i)} &&= (\lambda-1)t_i^2S_{yy}(i,i)e_ie_i^H\ttu^{(i)}.\\
\ea\ee
Noticing that the right hand side is a scaled version of the vector $e_1$, we have that 
\be
\ttu^{(i)} = \left(S_{wy}S_{yy}^{-1}S_{yw} -\lambda S_{ww} + (1-\lambda)t_i(e_iu_i^H + u_ie_i^H) \right)^{-1}e_i,
\ee 
but this vector may not be unit norm, and we divide by it's norm. Define
$\Psi(\lambda)=S_{wy}S_{yy}^{-1}S_{yw} -\lambda S_{ww}$. Then
\beq\label{eq:utt}
\left|\langle e_i, \ttu^{(i)}\rangle\right|^2
 = \frac{\left(e_i^H\left(\Psi(\lambda) + (1-\lambda)t_i(e_iu_i^H + u_ie_i^H) \right)^{-1}e_i\right)^2}{e_i^H\left(\Psi(\lambda) + (1-\lambda)t_i(e_iu_i^H + u_ie_i^H) \right)^{-2}e_i}.
\eeq
We notice that the matrix inverse is a rank-2 addition to $\Phi(\lambda)$ and so can be simplified
with the matrix inversion lemma. Also, define $\Phi(\lambda) = A^{-1}$ and $f(\lambda) =
e_i^H\Phi(\lambda)e_1$, $h(\lambda) = u_i^H\Phi(\lambda)u_i$ with $u_i = S_{wy}(:,i)$,
$q(\lambda)= e_i^H\Phi(\lambda)^2e_i$ and $s(\lambda)= u_i^H\Phi(\lambda)^2u_i$. From the
correlation derivation above, we recall that $e_1^H\Phi(\lambda)u_1=0$. With this notation and
the Woodubry inversion lemma, we have that the numerator is
\beq\label{eq:utt_num}
e_i^H\left(\Psi(\lambda) + (1-\lambda)t_i(e_iu_i^H + u_ie_i^H) \right)^{-1}e_i = \frac{f(\lambda)}{1-(1-\lambda)^2t_i^2f(\lambda)h(\lambda)}
\eeq
and that the denominator
\beq\label{eq:utt_den}\begin{split}
e_i^H&\left(\Psi + (1-\lambda)t_i(e_iu_i^H + u_ie_i^H) + \right)^{-2}e_i =  q(\lambda) +\\ &
\frac{2f(\lambda)h(\lambda)q(\lambda)}{\frac{1}{(1-\lambda)^2t_i^2} -
  f(\lambda)h(\lambda)} + \left(\frac{1}{(1-\lambda)^2t_i^2} -
  f(\lambda)h(\lambda)\right)^2\left(f(\lambda)^2h(\lambda)^2q(\lambda) +
  \frac{f(\lambda)^2s(\lambda)}{(1-\lambda)^2t_i^2}\right)
\end{split}
\eeq
We derived expressions for $f(\lambda)$ and $h(\lambda)$ in the correlation derivation,
specifically, (\ref{eq:f_lam}) and (\ref{eq:h_lam}). Therefore, we need expressions for
$q(\lambda)$ and $s(\lambda)$. Let's begin with $q(\lambda)$.

Similar to our correlation derivation, define the projection matrix 
\be
P_Y=Y^H(YY^H)^{-1}Y,
\ee
and matrices $B_1 = WPW^H$ and $B_2=W(I-P)W^H$. Then $\Psi= (1-\lambda)B_1 - 
\lambda B_2$. As discussed before, $B_1$ and $B_2$ are independent Wishart matrices. Then 
\be
q(\lambda) = e_1^H\left((1-\lambda)B_1 -\lambda B_2\right)^{-2}e_1.
\ee
Let $(1-\lambda)B_1 -\lambda B_2$ have limiting eigenvalue distribution
$\sigma(w)$. Then
\be
q(\lambda)\convas \int\frac{1}{x^2}d_{\mu_\Psi}(x).
\ee
Let $m_{\mu_\Psi}(z)$ be the Stieltjes transform of $\Psi$. Then we see that 
\be
m_{\mu_\Psi}(z) = \int\frac{1}{x-z}d_{\mu_\Psi}(x), \,\, m_{\mu_\Psi}^\prime(z) = -\int\frac{1}{(x-z)^2}d_{\mu_\Psi}(x)
\ee
Therefore, 
\be
q(\lambda)\to -m^\prime_{\mu_\Psi}(0).
\ee
To compute this, we use the R-transform trick that we employed in the correlation
derivation. First define $R_{b_1}(w)$ and $R_{b_2}(w)$ as the R-transforms of
$(1-\lambda)B_1$ and $-\lambda B_2$. From Bao et al. and above, we have
\be\ba
& R_{b_1}(w) = \frac{(1-\lambda)c_y}{1-(1-\lambda)c_xw}\\
& R_{b_2}(w) = \frac{-\lambda(1-c_y)}{1+\lambda c_xw}
\ea\ee\
By (\ref{eq:r_plus}), $R_\Psi = R_{b_1} + R_{b_2}$. 
Substituting $w = -m_{\mu_\Psi}(z)$ and using the relationship (\ref{eq:blue}), we have
\beq\label{eq:R_eigv}\ba
& R_{\Psi}(-m_{\mu_\Psi}(z)) && = R_{b_1}(-m_{\mu_\Psi}(z)) + R_{b_2}(-m_{\mu_\Psi}(z))\\
& z + \frac{1}{m_{\mu_\Psi}(z)} && = R_{b_1}(-m_{\mu_\Psi}(z)) + R_{b_2}(-m_{\mu_\Psi}(z)).
\ea\eeq
Plugging in the expressions for the individual R-transforms into (\ref{eq:R_eigv}), and doing
some algebra yields the following equality
\be\begin{split}
\lambda(1-\lambda)(c_x^2-c_x)m^2_{\mu_\Psi}(z) + &(cy-cx+2\lambda c_x-\lambda)m_{\mu_\Psi}(z) -1 =\\
&zm_{\mu_\Psi}(z)\left(1+c_xm_{\mu_\Psi}(z) -2\lambda
  c_xm_{\mu_\Psi}(z)-\lambda(1-\lambda)c_x^2m^2_{\mu_\Psi}(z)\right). 
\end{split}
\ee
Taking the derivative of both sides with respect to $z$, setting $z=0$ and solving for
$m_{\mu_\Psi}^\prime(0)$ yields 
\be
q(\lambda) = m_{\mu_\Psi}^\prime(0) = \frac{m_{\mu_\Psi}(0) + c_xm^2_{\mu_\Psi}(0)
  -2\lambda c_xm^2_{\mu_\Psi}(0)
  -\lambda(1-\lambda)c_x^2m^3_{\mu_\Psi}(0)}{\lambda(1-\lambda)(c_x^2-c_x)2m_{\mu_\Psi}(0) +
  c_y-c_x+2\lambda c_1-\lambda}.
\ee
We know from the correlation derivation that $m_{\mu_\Psi}(0) = f(\lambda)$, which completes
the derivation for $q(\lambda)$. 

A closed form expression for $s(\lambda)$ remains an open problem. However, substituting
empirically realizations of $\frac{1}{p}\Tr(e_i^H\Phi(\lambda)^2e_i)$ into
(\ref{eq:utt_den}) combined with the other closed form expressions to complete
(\ref{eq:utt_den}) and (\ref{eq:utt_num}) result in a good approximation of
(\ref{eq:utt}). This good approximation then can be used to solve for the canonical vector
accuracy in (\ref{eq:cca_wx_acc})

