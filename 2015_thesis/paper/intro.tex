Multi-modal data fusion is a ubiquitous problem in signal processing and machine
learning. In many applications, we have access to multiple datasets, possibly of different
modalities, each of which describe some feature of the system. This setup is becoming
increasingly common today as data collection becomes cheaper and easier. We are no longer
limited by the amount or variety of data that we can collect, but instead by how quickly
and accurately we can process such a wide variety of data.

The underlying assumption in such settings is that each dataset contains signals that are
correlated with signals of the other datasets. Correlation analysis algorithms hope to
leverage this fact to extrude these correlated signals \textit{jointly} from the datasets
more accurately than from the individual datasets alone. Of course, every application has
a different goal. Sometimes we want to detect the presence of the correlated
signals. Other times, we may wish to predict one modality from the other. In other
applications, we may desire to classify or cluster observations. Despite the differing
objectives, all of these applications rely on the ability to accurately detect and extract
the correlated signals between the datasets. This thesis focuses on developing
theoretically justified, robust correlations analysis algorithms to use as a
pre-processing step before learning algorithms that perform data fusion, as
motivated by Figure \ref{fig:data_fusion}.

\begin{figure}
\begin{center}  
    \begin{tikzpicture}[
      font=\sffamily, every matrix/.style={ampersand replacement=\&,column sep=3ex,row
      sep=3ex}, dataset/.style={draw,thick,fill=yellow!20,inner sep=.3cm},
      sink/.style={dataset,rounded corners,fill=black, text=white},
      app/.style={dataset,rounded corners,fill=blue!20}, dots/.style={gray,scale=2},
      to/.style={->,>=stealth',shorten >=1pt,semithick,font=\sffamily\footnotesize}, every
      node/.style={align=center}]

      \matrix{ \node[dataset] (dataset1) {Dataset 1}; \& \node[dataset] (dataset2)
        {Dataset 2}; \& \node[dataset] (dataset3) {Dataset 3}; \\

        \& \node[sink] (blackbox) {Correlation Analysis}; \& \\

        \& \node[app] (application) {Learning Algorithm}; \& \\ };

      \draw[to] (dataset1) -- (blackbox); \draw[to] (dataset2) -- (blackbox); \draw[to]
      (dataset3) -- (blackbox); \draw[to] (blackbox) -- (application);

    \end{tikzpicture}
    \caption{Illustration of multi-modal data fusion}
    \label{fig:data_fusion}
\end{center}
\end{figure}


\section{Canonical Correlation Analysis (CCA)}

\subsection{What is it? What is it not?}

Canonical correlation analysis (CCA) is a joint dimensionality reduction algorithm for
exactly two datasets that finds a linear transformation for each dataset such that the
correlation between the two transformed feature sets is maximized
\cite{hotelling1936relations}. CCA, however, \textit{is not} a data fusion algorithm. CCA
returns two linear transformations and a set of correlations. In this light, CCA is
extremely similar to principle component analysis (PCA), which returns a linear
transformation that accounts for the directions of largest possible variance in a
dataset. These principle components are typically used as features vectors in a variety of
machine learning algorithms. Just as PCA is a dimensionality reduction algorithm and not
the final machine learning algorithm that uses the principle components, CCA is a joint
dimensionality reduction algorithm whose dimensionality reduction ensures that datasets
are maximally correlated in their reduced spaces. These maximally correlated features may
then be used however a learning algorithm desires. 

The solution to CCA is easily found by solving a quadratic optimization problem. This
solution is a closed form expression relying on the singular value decomposition (SVD) of
a matrix product involving the covariance matrices of each dataset and the
cross-covariance between the two datasets. As these covariance matrices are rarely known
\textit{a priori}, practical uses of CCA rely on substituting sample covariance matrices
formed from training data, which we call empirical CCA.

The performance of empirical CCA has been studied previously, but insufficiently. When the
number of training samples is large compared to the dimensions of the datasets, the
performance is well understood \cite{gunderson1997estimating}. When the number of training
samples is less than the sum of the dimension of each dataset (sample deficient regime),
\cite{pezeshki2004empirical} proves that empirical CCA completely breaks down and always
reports a perfect correlation between the datasets. 

This extremely undesirable characteristic of empirical CCA has lead many to abandon CCA as
a reliable statistical analysis technique. Pezeshki, L.L. Scharf et al. argue that in this
sample deficient regime
\begin{quote}
  ... the \textcolor{red}{empirical canonical correlations are defective and
    may not be used} as     estimates of canonical correlations between random
  variables.\cite{pezeshki2004empirical}
\end{quote}
Similarly, Ge et al. conclude that
\begin{quote}
... CCA provide(s) \textcolor{red}{reliable information} about spatial
    correlations existing among pairs of data sets \textcolor{red}{only when SNRs ... are
    reasonably high, and the sample support is significantly larger than the data
    dimensions.}\cite{ge2009does}
\end{quote}



\subsection{Variations on CCA}

Due to this undesirable breakdown of CCA in the low-sample high-dimensionality regime,
many researchers proposed variations of CCA to avoid this performance loss. Most notably,
\cite{nadakuditi2011fundamental} used recent results from random matrix theory to
demonstrate that this performance breakdown may be avoided by trimming the sample
covariance matrix estimates to only include informative components. This algorithm is the
crux of this thesis. We will study its performance and develop theoretical tools in order
to use it for real-world applications. Throughout the thesis, we use the ubiquitous
low-rank signal-plus-noise model for datasets \be X = UV^H + Z, \ee where
$X=[x_1,\dots,x_n]$ is our observed data matrix whose columns are individual
multidimensional observations, $U$ is a low-rank signal subspace, $V$ is a low-rank signal
matrix, and $Z$ is a noise matrix. Surprisingly, correlation analysis for this classical
low-rank signal-plus-noise model is not completely studied. This thesis seeks to complete
the discussion. Here, we briefly touch on other variations based on CCA that do not assume
the above linear low-rank signal-plus-noise model. Many of these algorithms are tuned for
a specific application or seek to avoid the performance loss of CCA in a certain regime.

Regularized CCA (RCCA) \cite{vinod1976canonical} adds a penalty term to the magnitude of
the canonical vectors. This results in adding a scaled copy of the identity matrix to the
sample covariance matrix of each dataset, which allows each matrix to be
inverted. Therefore, RCCA returns non-trivial results in the sample-deficient
regime. However, this approach introduces a parameter to the algorithm; the effect of this
parameter is not well studied. Other variations of RCCA, such as supervised
RCCA \cite{thum2014supervised}, fast RCCA \cite{cruz2014fast}, and a multi-block RCCA
\cite{tenenhaus2014regularized}, have also been proposed.

Kernel CCA (KCCA) \cite{akaho2006kernel} was proposed to deal with non-linear correlations
existing between datasets. However, KCCA also introduces regularization parameters so as
to not return trivial solutions (see \cite{welling2005kcca} for an excellent
derivation). Besides the choice of regularization parameter, there is also ambiguity in
the choice of the kernel function, which is a common problem among kernel methods. Other
variations of KCCA have also been proposed, such as penalized KCCA
\cite{waaijenborg2009correlating}, alpha-beta divergence \cite{mandal2013non}, and CCA
based on kernel target alignment \cite{chang2013canonical}. 

Sparse CCA \cite{hardoon2011sparse} finds linear transformations such that the number of
features used is minimized. This problem is often motivated by the need for interpretable
canonical vectors that is often driven by the application, such as in brain imaging
\cite{yan2014accelerating}. There are many variations on sparse CCA, typically motivated
by application or mathematical intrigue. Sun and Keates \cite{sun2013canonical} explore
CCA in the context of censoring, Shin and Lee examine sparse functional data
\cite{shin2015canonical}, Tao et al. consider joint sparse data in
\cite{tao2014exploring}, Gao et al. explore efficient sparse CCA for high-dimensional
data \cite{gao2014efficient}, and Zhang et al. extend the analysis to multi-class group
sparse CCA \cite{zhang2013binary}. Other formulations include a penalized decomposition
\cite{witten2009penalized}, Bayesian CCA via group sparsity \cite{klami2013bayesian}, and
recursive sparse CCA \cite{chu2013sparse}.

\subsection{Applications}

CCA and its variants are widely used in a variety of fields where multiple datasets
naturally arise, the most common of which is machine learning and computer vision. In
\cite{hardoon2004canonical}, CCA is used to learn semantics of multimedia content by
fusing image and text data. Related, \cite{dhillon2011multi} uses CCA to learn word
embeddings for supervised natural language processing tasks. CCA has been widely applied
to pose estimation \cite{melzer2001nonlinear,zhai2015instance}, as this is a natural
examples where we have multiple views (image) of the same object. Other computer vision
related tasks where correlation methods are natural fits include matching people across
cameras \cite{lisanti2014matching}, clustering social event images
\cite{ahsan2014clustering}, automatic image annotation \cite{hardoon2006correlation}, and
audio-visual speaker clustering \cite{chaudhuri2009multi}.

Medical analysis is another field where there are ripe opportunities for correlation
analysis due to the vast number of modalities (EEG, MRI, CT, fMRI, MEG, etc.). CCA is
often used to determine interactions, or connectivities, between brain areas in fMRI data
\cite{deleus2011functional,arbabshirani2010comparison,khalid2013improving,guccione2013functional}
and used to fuse fMRI, sMRI, and EEG data \cite{correa2010canonical}. CCA based methods
have also been used to examine genetic connections
\cite{lin2013identifying,seoane2014canonical,lin2013group}, relying heavily on sparse
methods due to the high dimensionality of gene data and need to interpret which genes are
``on''. CCA is also a popular way to detect frequencies in steady-state visual evoked
potential (SSVEP) in brain-computer interfaces (BCIs)
\cite{zhang2013l1,nakanishi2014enhancing,zhang2014frequency}. Still further, CCA is used
in de-noising and analysis of EEG, MEG and ECG data
\cite{spuler2013spatial,campi2013non,chen2014removal,kuzilek2014comparison}.

CCA also has roots in classical signal processing applications. The authors of
\cite{via2005canonical} apply CCA to to the common communications problem of blind
equalization of single-input multiple-output (SIMO) channels. Pezeshki et al.
\cite{pezeshki2006canonical} showed that the CCA coordinates are the correct coordinates
for low-rank Gauss-Gauss detection and estimation. Scharf and Thomas
\cite{scharf1998wiener} provide a wonderful exposition on using the canonical coordinates
for Wiener filters, transform coding, filtering, and quantizing. CCA and multiset CCA have
been used to achieve joint blind source separation (BSS) in \cite{li2009joint}. CCA has
also been applied to hyperspectral imaging \cite{nielsen2002multiset}, array processing
\cite{ge2009does}, Gaussian channel capacity \cite{scharf2000canonical}, and cognitive
radio networks \cite{manco2014kernel}.

Other fun and interesting applications include climatology, finance, and music. Todros and
Hero define a new measure transformed based CCA and show its utility on financial data in
\cite{todros2012measure}. Torres et al. \cite{torres2007finding} use sparse CCA to label
portions of musical songs with meaningful words or phrases. In the field of climatology,
CCA has been used to study sea temperatures \cite{wilks2014probabilistic}, forest planning
\cite{prera2014using}, and tropical cyclones \cite{steward2014assimilating}. Finally, I
would be remiss if I didn't share my personal favorite application of CCA to date: using
CCA to analyze bovine growth \cite{li2010canonical}.

\section{Contributions of this thesis}

In many of the application presented above, researchers either have access to many
samples, or have designed an algorithm tuned specifically for a their particular
application. This thesis considers the performance of empirical correlation algorithms in
the low-sample, high dimensional setting. These algorithms are not geared toward a
particular application but are general and may be applied to any application. We will
demonstrate, both theoretically and empirically, that multi-modal correlation analysis in
this regime is a possibility. We remark that statements labeled as theorems represent, to
the best of our knowledge, new results while important results from literature are labeled
as propositions or lemmas. Chapters II-III are self contained and may be read
independently. Chapters IV-X consider the problem of correlation analysis.

Chapters \ref{sec:chpt_msd} and \ref{sec:chpt_msd_exten} consider the classical problem of
matched subspace detectors. We use insights from random matrix theory on the accuracy of
subspace estimates to derive new, optimal detectors that demonstrate the sub-optimality of
the classical plug-in detectors that simply substitute maximum likelihood estimates for
unknown parameters. Under both a stochastic and deterministic data model, we argue that
only the \textit{informative} subspace components should be used in a detector. We extend
this analysis to the case where our observations may contain missing data.

In Chapters \ref{sec:chpt_cca_det} and \ref{sec:chpt_cca_vects}, we explore the
performance of CCA and re-derive informative CCA (ICCA). We demonstrate the extreme
sub-optimality of CCA in the low-sample, high dimensionality regime. Specifically, we
provide a statistical test for both CCA and ICCA that determines whether the correlations
returned by the algorithms do indeed represent a true underlying correlation in the
datasets. We prove when each of these statistics are consistent to showcase the
superiority of ICCA. We also provide an analogous statistical test and consistency theorem
to use when the datasets have missing data entries. We create 3 new real-world,
multi-modal datasets involving video and audio to verify the performance of ICCA. We then
showcase that the canonical vectors returned by ICCA are more accurate than the CCA
vectors in the low-sample regime. Finally, we provide a new algorithm for estimating the
canonical vectors that asymptotically optimal.

Next, we explore the performance of regularized CCA (RCCA) in Chapter
\ref{sec:chpt_rcca}. When the number of training samples is limited but correlation
analysis is still desired, a common strategy is to regularize CCA by adding a penalty to
the magnitude of the linear transformation. However, we demonstrate that setting the
regularization parameter to infinity results in the best performance. In fact, in this
setting, the solution to RCCA may be found by taking the SVD of the sample
cross-covariance matrix between the two datasets. We then predict the behavior of the
largest singular values of this cross-covariance matrix assuming a low-rank
signal-plus-noise model on the individual datasets. We argue that using the top singular
values of this cross-covariance matrix to detect correlations is sub-optimal because the
correlation coefficients are coupled with the individual data signal strengths.

Using a similar proof technique, we predict the behavior of the largest singular values of
the projection of low-rank signal plus noise matrices to a smaller dimension in Chapter
\ref{sec:chpt_svd_proj}. Specifically, we consider two types of projection matrices: one
with standard complex Gaussian entries and one with orthonormal columns. We are able to
provide a closed form expression for the largest singular values in the case where the
projection matrix is unitary. Through numerical simulations, we demonstrate the
superiority of the unitary projection matrix over the Gaussian projection matrix. The
unitary projection matrix can reliably detect the signals at a lower signal-to-noise ratio
than the Gaussian projection matrix. 

In Chapter \ref{sec:chpt_det_reg} we apply CCA and ICCA to the classical problems of
detection and regression. First, we consider the low-rank signal-versus-noise subspace
detection problem given two datasets. We prove that the standard likelihood ratio test
(LRT) detector may be written using the canonical basis returned by CCA. We show that when
using empirical parameter estimates, the CCA detector is extremely suboptimal but that the
ICCA detector is equivalent to the plug-in LRT detector. We then show that the classical
Gaussian regression problem may be written in terms of the CCA basis. However, similar to
the detection problem, empirical CCA degrades the performance significantly while ICCA
matches the classical plug-in detector. We show this via mean squared error prediction
plots.

We then consider the joint problems of image retrieval and image annotation in Chapter
\ref{sec:chpt_ia}. Correlation based methods are typically overlooked as solutions to such
problems due to the problems with CCA outlined in this thesis. We show that using ICCA to
solve these problems results in non-trivial solutions. We compare the performance of CCA
and ICCA on four different image-text datasets and describe the capabilities and
limitations of ICCA in this application. When the datasets contain multiple images of the
same objects and meaningful captions, ICCA is able to capture correlations between images
and text. However, ICCA fails to capture semantic meanings between documents and
captions. We argue that with clever feature engineering and improved NLP techniques,
correlation based methods may be relevant for image retrieval and image annotation.

Lastly, we consider multi-set CCA (MCCA) in Chapter \ref{sec:chpt_mcca}. Unfortunately,
unlike CCA, there is no clear objective function to use in an optimization problem;
Kettenring \cite{kettenring1971canonical} proposes five such objective functions. Nielsen
also provides a nice formulation of MCCA in \cite{nielsen1994analysis} where he proposes
four constraint functions.  We provide derivations for these 20 formulations, in both a
theoretical and empirical setting. We then choose to consider the MAXVAR problem as we are
able to directly apply our insights from ICCA to create an informative version of it,
which we call informative MCCA (IMCCA). We demonstrate the superior performance of IMCCA
on a real-world video dataset that we created. 
