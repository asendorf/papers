A ubiquitous problem in signal and array processing is designing multi-dimensional
signal-plus-noise versus noise detectors. In such applications, an observation
$w$ may belong to either the noise only hypothesis ($H_0$) or the signal-plus-noise hypothesis
($H_1$), via the model
\beq\label{eq:general_setup} 
w = \begin{cases}
  z & w\in H_0\\
  \delta + z & w\in H_1,\\
\end{cases}
\eeq
where $\delta$ is the unknown signal vector and $z$ is additive noise. When modeling
$\delta$ as a fixed deterministic vector and $z$ as Gaussian noise, the standard detector
statistic is $\|w\|^2$, the squared norm (magnitude) of the observed vector $w$. This
detector is commonly referred to as an energy detector because the squared norm measures
the amount of energy contained in the observation. Energy detectors arise in applications such
as incoherent radar detection \cite{cui2013performance}, Global Navigation Satellite
Systems (GNSS) \cite{arribas2013antenna}, and MIMO radar
\cite{gorji2013widely,zhou2013space}.

In this chapter, we analyze the performance of the energy detector, starting from first
principles. Using a receiver operating characteristic (ROC) performance analysis, we
investigate the conditional distributions of the energy detector's test statistic and
showcase how these distributions shift depending on the number of signal components that
the energy detector uses. We saw in the previous chapter that including more than the
$\keff$ number of subspace components degrades detector performance. In this chapter we
show that, surprisingly, even if a signal component is one of the $\keff$ informative
components, if its signal strength is too small,
including it in an energy detector actually degrades detector performance. Using this
observation, we define the number of signal components that maximize detector performance
as $\kuse$, which is dependent on the desired false alarm rate of the energy detector. Our
goal is to bring this phenomenon into focus so that effort can be spent on designing
better real world detectors.

We are motivated by the more specific problem of deterministic matched subspace
detection. A matched subspace detector (MSD) is commonly used to detect a signal buried in
high dimensional noise under the assumption that the signal lies in a low-rank signal
subspace. Many applications in signal and array processing use such low-rank
signal-plus-noise models, including incoherent radar detectors \cite{cui2013performance},
direction detection \cite{santiago2013noise,hu2013doa,liao2013direction}, GNSS
\cite{arribas2013antenna}, MIMO radar
\cite{chen2013adaptive,gorji2013widely,zhou2013space} and target detection
\cite{kwon2013multi}. A deterministic signal model, which assumes that the target signal
lies at an unknown but fixed point in the signal subspace, occurs in array processing
\cite{besson2005matched,bandiera2007glrt,bandiera2007adaptive}, MIMO radar
\cite{sirianunpiboon2013multiple}, and cognitive radio \cite{vazquez2011spatial}.
When the signal subspace is known \textit{a priori}, the performance of such deterministic
MSDs has been extensively studied (see, for example,
\cite{scharf1994matched,vincent2008matched,besson2006cfar}). In a recent paper
\cite{asendorf2013performance}, we considered the performance of a MSD in the alternative
setting where the signal subspace is unknown and estimated from finite, noisy,
signal-bearing training data.

Under a deterministic signal model and appropriate noise assumptions, a MSD is an energy
detector that projects a observation onto this estimated signal subspace and uses the
squared norm of the projection as the detector's statistic. In
\cite{asendorf2013performance}, we used random matrix theory (RMT) to showcase that using
more than the $\keff$ \textit{informative} subspace components decreases detector
performance. In this chapter, we show that even though a subspace component may be informative (as
defined by $\keff$), including it in a detector may degrade performance. Using exactly the
$\kuse$ subspace components results in the best detector performance. However, as $\kuse$
is computed assuming knowledge of the unknown deterministic vector, $\keff$ provides a
realizable upper bound for $\kuse$.

Finally, we consider the deterministic MSD setting where the training data
is noisy \textit{and} has missing entries. The missing entry context is motivated in
\cite{balzano2010high} by distributed detection scenarios where it might be prohibitive to
collect and transmit only a (randomly chosen) fraction $p$ of the training data
entries. Alternately one might think of $1-p \in (0,1)$ as a compression factor as in
compressed sensing. We precisely quantify the performance of the MSD with missing data. We
uncover a phase transition phenomenon by showing that there is 
a critical fraction, $p_\text{crit}$, which is a simple function of the eigen-SNR, the
number of training samples, and the number of sensors, below which detection performance
deteriorates to random guessing. Compressing the training dataset below this critical
fraction is undesirable.

The chapter is organized as follows. In Section \ref{sec:energy_detector}, we formulate
the standard signal versus noise detection problem and derive the standard energy
detector. We discuss the energy detector's conditional distributions, define $\kuse$, and
discuss its properties in Section \ref{sec:useful}. In Section \ref{sec:msd}, we apply
these insights to deterministic MSDs and highlight the relationship between $\keff$ and
$\kuse$ through numerical simulations. In Section \ref{sec:ext}, we discuss the weighted
energy detector as a natural extension to this work.  We extend the results to the setting
where our original data matrix may have missing data in Section
\ref{sec:chpt3:missing}. Finally, we provide concluding remarks in Section
\ref{sec:conclusion}.
