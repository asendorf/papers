We wish to design a detector that discriminates between the $H_0$ hypothesis that an
observation is purely noise and the $H_1$ hypothesis that the observation contains an
unknown signal.  We model the observation $w\in\reals^{k\times 1}$ as in
(\ref{eq:general_setup}) where $\delta=[\delta_1,\dots,\delta_k]^T\in\reals^{k\times 1}$,
with $\delta_i\neq 0$, is an unknown deterministic vector,
$z\sim\mathcal{N}\left(0,I_k\right)$ is additive white Gaussian noise (AWGN), and $k$ is
known. See
\cite{cui2013performance,santiago2013noise,arribas2013antenna,chen2013adaptive,gorji2013widely,zhou2013space,kwon2013multi,hu2013doa,liao2013direction}
for similar signal-plus-noise models in signal and array processing.  In the
Neyman-Pearson detection setting (see \cite{van1968detection}), the detector for this data
model is the likelihood ratio test (LRT)
\beq\label{eq:lrt_form}
\Lambda(w) = \frac{f\left(w\,|\, H_1\right)}{f\left(w\,|\, H_0\right)} \detgtrless \eta.
\eeq
Here $f\left(\cdot\right)$ is the appropriate conditional probability density function
(p.d.f.) of the observation and $\eta$ is a scalar threshold set so that $\Prob{\Lambda(w)
  > \eta\,|\, w\in H_0} = \alpha$ where $\alpha\in[0,1]$ is a desired false alarm rate.

The conditional distributions of $w$ modeled as in (\ref{eq:general_setup}) are
$w|H_0\sim\mathcal{N}\left(0,I_k\right)$ and
$w|H_1\sim\mathcal{N}\left(\delta,I_k\right)$. However, as $\delta$ is unknown, we cannot
substitute the p.d.f. of $w|H_1$ into (\ref{eq:lrt_form}). Instead, we use the generalized
LRT (GLRT), which maximizes $f(w|H_1)$ with respect to any unknown parameters. The GLRT
for our problem is 
\be \Lambda(w) = \frac{\max_{\delta}f\left(w\,|\,
    H_1\right)}{f\left(w\,|\, H_0\right)} \detgtrless \eta.  
\ee
The conditional p.d.f. of $w$ under the $H_1$ hypothesis is
\be
f\left(w\,|\,H_1\right) = \left(2\pi\right)^{-k/2}\exp\left\{ -\frac{1}{2}\left(w-\delta\right)^T\left(w-\delta\right)\right\}.
\ee
This p.d.f. is maximized when $\delta=w$ with the maximum value of
$(2\pi)^{-k/2}$. Substituting this into the GLRT yields
\be
\Lambda(w) = \exp\{\frac{1}{2}w^Tw\}
\ee
Taking the natural logarithm results in the test statistic
\beq\label{eq:energy_detector}
\Lambda_{\text{energy}}(w) = w^Tw = \sum_{i=1}^k w_i^2
\eeq
where $w=[w_1,\dots,w_k]^T$. This is an energy detector as its test statistic sums the
energy residing in each component (or dimension) of the given observation.

\subsection{ROC Curve Analysis}

To compare the performance of multiple detectors, we will compare their receiver operating
characteristic (ROC) curves. A ROC curve is a collection of points ($P_F,P_D$) where for
$-\infty<\eta<\infty$,
\beq\label{eq:roc}\ba
&P_F=\Prob{\Lambda(w) > \eta \,|\, w\in H_0},\\
&P_D=\Prob{\Lambda(w) > \eta \,|\, w\in H_1}.\\
\ea\eeq
For $0\leq P_F\leq 1$ we want to express the probability of detection $P_D$ as a function
of the false alarm rate, $P_F$, while noting that $P_F$ is a function of
$\eta$. To make analytical progress, we assume that $\delta$ is known for ROC
derivations. First, we compute the conditional distributions of the statistic in
(\ref{eq:energy_detector}). The conditional distributions of the components in $w$ are
simply 
$w_i|H_0\overset{\text{i.i.d.}}{\sim}\mathcal{N}\left(0,1\right)$ and
$w_i|H_1\overset{\text{i.i.d.}}{\sim}\mathcal{N}\left(\delta_i,1\right)$. Therefore,
$w_i^2|H_0\overset{\text{i.i.d.}}{\sim}\chi^2_1$ and
$w_i^2|H_1\overset{\text{i.i.d.}}{\sim}\chi^2_1\left(\delta_i^2\right)$ where $\chi^2_1$
is a chi-square random variable with one degree of freedom and $\chi^2_1(\delta_i^2)$ is a
non-central chi-square random variable with one degree of freedom and non-centrality
parameter $\delta_i^2$. As each component $w_i$ is independent,
\beq\label{eq:distr_energy} \ba
&\Lambda(w)|H_0\sim\chi^2_k,\\
&\Lambda(w)|H_1\sim\chi^2_k\left(\delta^T\delta\right), 
\ea \eeq 
where $\chi^2_k$ is a chi-square random variable with $k$ degrees of freedom and $\chi^2_k(\delta^T\delta)$ is a
non-central chi-square random variable with $k$ degrees of freedom and non-centrality
parameter $\delta^T\delta = \sum_{i=1}^k\delta_i^2$. Armed with these characterizations in
(\ref{eq:distr_energy}) and solving for $\eta$ in (\ref{eq:roc}), we can relate $P_D$ to
$P_F$ using the expression
\begin{equation}\label{eq:roc_energy}
\begin{aligned}
&P_{D_\text{energy}}(P_F,k)=1-Q_{\chi_{k}^2(\lambda_k)}\left(Q^{-1}_{\chi^2_{k}}\left(1-P_F\right)\right).\\
\end{aligned}
\end{equation}
In (\ref{eq:roc_energy}), $Q_{\chi_{k}^2}(\lambda_k)$ is the cumulative distribution function (c.d.f.) of a
non-central chi-square random variable with $k$ degrees of freedom and non-centrality
parameter $\lambda_k=\sum_{i=1}^k\delta_i^2$ and $Q_{\chi_{k}^2}$ is the c.d.f. of a chi-square random
variable with $k$ degrees of freedom. See \cite{cui2013performance,arribas2013antenna,zhou2013space} for similar ROC
performance curve derivations.

\subsection{Problem Statement}\label{sec:prob_state}

As practitioners, we can control which signal components that the energy detector in
(\ref{eq:energy_detector}) uses. Without loss of generality, we assume that the entries of
$\delta$ are ordered (i.e. $|\delta_1| \geq |\delta_2| \geq\dots|\delta_k|$). With this
assumption, we can decide how many signal components, $d$, to use in the energy detector
\beq\label{eq:the_detector} 
\Lambda_d(w) = \sum_{i=1}^d w_i^2.
\eeq 
Specifically, we wish to answer the following question:
\begin{quote}
  Given a signal vector $\delta$ and a desired false alarm rate $P_F$, how many signal
  components, $d$, maximize $P_{D_{\text{energy}}}(P_F,d)$ in (\ref{eq:roc_energy}) for an
  energy detector with the form of (\ref{eq:the_detector}) derived from observations
  as in (\ref{eq:general_setup})? 
\end{quote}
Answering this question will provide some surprising results. We will show that if the
components $\delta_i$ are too small in magnitude, including them in a detector actually
degrades performance. The setting where $\delta_i$ equals zero is a special case where not
including it will always yield a performance gain.