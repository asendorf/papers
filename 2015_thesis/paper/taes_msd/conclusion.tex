In this chapter, we considered the problem of designing a signal-plus-noise versus noise
detector when the signal is assumed to be a fixed deterministic vector. In such a setting,
the GLRT detector is an energy detector, whose statistic is the squared norm of the
observation. By examining how the conditional distributions of this test statistic shift
when adding additional components, we derived and defined the number of useful components,
$\kuse$, that maximize detection ability.

When adding a component to an energy detector, there is a tradeoff between increasing the
noise variance and increasing the signal variance by increasing the non centrality
parameter. For a component to be one of the $\kuse$ components, the increase in
non-centrality parameter must overcome the added noise variance. We explored the necessary
increase in non-centrality parameter needed for a component to be useful in Figure
\ref{fig:nc_lines}.

We applied the idea of using only $\kuse$ components to deterministic matched subspace
detection where the unknown signal subspace is estimated from finite, noisy,
signal-bearing training data. Both the standard plug-in detector using $k$ subpsace
components and RMT detector using $\keff$ subspace components (as defined by
(\ref{eq:keff})) are energy detectors. We demonstrated that the new useful subspace
detector outperforms both the plug-in and RMT detectors. Importantly, we showed that while
a subspace component may be informative ($|\langle u_i,\widehat{u}_i\rangle|^2 >0$), using
that component in a detector may decrease performance.

As detectors using $\kuse$ components assume knowledge of the unknown signal vector, they
are not realizable. We showed that $\keff$ may be used as an upper bound for $\kuse$,
however, deriving other estimates for $\kuse$ that can be used in applications other than
matched subspace detection is a focus of future work. We also provided a disucssion about
the more general weighted energy detector and showed that such a detector can improve
detection performance. Determining an efficient algorithm to compute the optimal weighting
matrix to use in the weighted energy detector is an important area of future
research. Extending the performance analysis of the useful matched subspace detector to
the case of unknown $\Sigma$ or complex valued data is within reach. The work in
\cite{asendorf2013performance} on eigen-SNR accuracy, estimating $\keff$, and estimating
$|\langle u_i,\widehat{u}_i\rangle|^2$ is directly applicable.

Finally, we considered a deterministic MSD problem where the unknown low-rank signal subspace is
estimated from noisy, limited, signal-bearing training data with missing entries.  We used
RMT to characterize the resulting performance and showed that using $k_\text{eff} \leq k$
subspace components is optimal. The relationship between $k_{\text{eff}}$ and $p$ was made
explicit in Theorem \ref{th:angles} and we showed that detection better than random
guessing (in the large system limit) is only achievable for $p > p_{\text{crit.}} : =
\sqrt{c}/\max_{i}(\sigma_{i}^{2})$.
