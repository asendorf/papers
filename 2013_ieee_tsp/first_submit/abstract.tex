We analyze the performance of a matched subspace detector (MSD) where the test signal vector is assumed to reside in an unknown, low-rank $k$ subspace which must be estimated from finite, noisy, signal-bearing training data. We consider stochastic and deterministic models for the test vector and characterize the associated ROC performance curves of the resulting detectors. Our analysis utilizes recent results from random matrix theory (RMT) that precisely quantify the quality of the subspace estimate as a function of the eigen-SNR, dimensionality of the system, and the number of training samples. We exploit this knowledge of the subspace estimation accuracy to derive a new RMT detector whose performance matches that of a plug-in detector that uses only the $k_\text{eff} \leq k$  `informative' eigenvectors of the sample covariance matrix. Detectors using more than $k_\text{eff}$ components will realize a performance loss that our analysis quantifies in the large system limit.  We validate our asymptotic predictions with simulations on moderately sized systems.

%We consider both the case where the signal is placed randomly in the unknown subspace and the case where the signal is placed deterministically at an unknown location in the unknown subspace. In the high-dimensional limited-sample setting or in regimes of moderate to low signal-to-noise ratio (SNR), the subspace estimate is inaccurate. We use random matrix theory (RMT) to precisely quantify these subspace estimation errors and characterize the ROC performance of detectors whose test statistic is of the form $w^HDw$. One such detector, the standard plug-in detectors, assumes that the subspace estimate is exact. We show that, in certain regimes, this assumption results in a performance loss. Using the subspace estimation accuracy provided by RMT, we derive a new RMT detector which avoids the performance losses associated with the plug-in detector. Using numerical simulations, we verify our ROC predictions, validate the accuracy of the RMT approximations, and highlight the optimality of using the $k_\text{eff}$ \textit{informative} signal subspace components.
