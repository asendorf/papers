\textit{Theorem 5.1:} Assume the same hypothesis as in Proposition \ref{th:angles}. Let $\widehat{k}=\keff=k$. For $i=1,\dots,\widehat{k}$, $j=1,\dots,k$, and $i\neq j$, as $n,m\to\infty$ with $n/m\to c$, then $\langle u_j,\widehat{u}_i\rangle \convas 0.$
\begin{proof}
Let $U_{n,k}$ be a $n\times k$ real or complex matrix with orthonormal columns, $u_i$ for $1\leq i\leq k$. Let $\Sigma = \diag\left(\sigma_1^2,\dots,\sigma_k^2\right)$ such that $\sigma_1^2>\sigma_2^2>\dots>\sigma_k^2>0$ for $k\geq 1$. Define $P_n=U_{n,k}\Sigma U_{n,k}^H$ so that $P_n$ is rank-$k$. Let $Z_n$ be a $n\times m$ real or complex matrix with independent $\mathcal{CN}\left(0,1\right)$ entries. Let $X_n=\frac{1}{m}Z_nZ_n^H$, which is a random Wishart matrix, have eigenvalues $\lambda_1(X_n)\geq\dots\geq\lambda_n(X_n)$. Let $\widehat{X}_n=X_n\left(I_n+P_n\right)$. $X_n$ and $P_n$ are independent by assumption. Define the empirical eigenvalue distribution as $\mu_{X_n}=\frac{1}{n}\sum_{j=1}^n\delta_{\lambda_j\left(X_n\right)}$. We assume that as $n\to\infty$, $\mu_{X_n}\overset{\text{a.s.}}{\longrightarrow}\mu_X$.


For $i=1,\dots, \widehat{k}=k$, let $\widehat{v}_i$ be an arbitrary unit eigenvector of $\widehat{X}_n$. By the eigenvalue master equation, $\widehat{X}_n\widehat{v}_i=\widehat{\lambda}_i\widehat{v}_i$, it follows that
\begin{equation}\label{eq:eval_master}
\begin{aligned}
  &U_{n,k}^H\left(\widehat{\lambda}_iI_n-X_n\right)^{-1}X_nU_{n,k}\Sigma U_{n,k}^H\widehat{v}_i&&=U_{n,k}^H\widehat{v}_i.\\
\end{aligned}
\end{equation}
Let $X_n=V_n\Lambda_nV_n^H$ be the eigenvalue decomposition of $X_n$ such that $\Lambda_n=\diag(\lambda_1(X_n),\dots,\lambda_n(X_n))$ and $\lambda_1(X_n)\geq\dots\geq\lambda_n(X_n)$. Using this decomposition and defining $W_{n,k}=V^HU_{n,k}$, (\ref{eq:eval_master}) simplifies to
\begin{equation}\label{eq:eval_master2}
\begin{aligned}
  &W_{n,k}^H\left(\widehat{\lambda}_iI_n-\Lambda_n\right)^{-1}\Lambda_nW_{n,k}\Sigma U_{n,k}^H\widehat{v}_i&&=U_{n,k}^H\widehat{v}_i.\\
\end{aligned}
\end{equation}
Define the columns of $W_{n,k}$ to be $w_j^{(n)}=[w_{1,j}^{(n)},\dots,w_{n,j}^{(n)}]^T$ for $j=1,\dots,k$. These columns are orthonormal and isotropically random. We can rewrite (\ref{eq:eval_master2}) as
\begin{equation}\label{eq:t_trans}
\left[T_{\mu_{r,j}^{\left(n\right)}}\left(\widehat{\lambda}_i\right)\right]_{r,j=1}^k \Sigma U_{n,k}^H\widehat{v}_i=U_{n,k}^H\widehat{v}_i
\end{equation}
where for $r=1,\dots,k$, $j=1,\dots,k$, $\mu_{r,j}^{\left(n\right)}=\sum_{\ell=1}^n\overline{w_{\ell,r}^{\left(n\right)}}w_{\ell,j}^{\left(n\right)}\delta_{\lambda_\ell\left(X_n\right)}$ is a complex measure and $T_{\mu_{r,j}^{\left(n\right)}}$ is the T-transform defined by $T_{\mu}\left(z\right) = \int\frac{t}{z-t}d\mu\left(t\right)\,\,\,\,\text{for } z\not\in\text{supp } \mu$. We may rewrite (\ref{eq:t_trans}) as
\begin{equation*}
\left(I_k-\left[\sigma_j^2T_{\mu_{r,j}^{\left(n\right)}}\left(\widehat{\lambda}_i\right)\right]_{r,j=1}^k\right)U_{n,k}^H\widehat{v}_i=0.
\end{equation*}
Therefore, $U_{n,k}^H\widehat{v}_i$ must be in the kernel of $M_n\left(\widehat{\lambda}_i\right)=I_k-\left[\sigma_j^2T_{\mu_{r,j}^{\left(n\right)}}\left(\widehat{\lambda}_i\right)\right]_{r,j=1}^k$.
By Proposition 9.3 of \cite{benaych2011eigenvalues}
\begin{equation*}
\mu_{r,j}^{\left(n\right)}\overset{\text{a.s.}}{\longrightarrow}\begin{cases}\mu_X & \text{for } i=j \\ \delta_0 & \text{o.w.} \end{cases}
\end{equation*}
where $\mu_X$ is the limiting eigenvalue distribution of $X_n$. Therefore,
\begin{equation*}
M_n\left(\widehat{\lambda}_i\right)\overset{\text{a.s.}}{\longrightarrow}\diag\left(1-\sigma_1^2T_{\mu_X}\left(\widehat{\lambda}_i\right), \dots, 1-\sigma_k^2T_{\mu_X}\left(\widehat{\lambda}_i\right)\right).
\end{equation*}
As $k_\text{eff}=k$, for $i=1,\dots,k$, $\sigma_i^2>1/T_{\mu_X}(b^+)$, where $b$ is the supremum of the support of $\mu_X$. As $\widehat{\lambda}_i$ is the eigenvalue corresponding to the eigenvector $\widehat{v}_i$, by Theorem 2.6 of \cite{benaych2011eigenvalues} $\widehat{\lambda}_i\overset{\text{a.s.}}{\longrightarrow}T^{-1}_{\mu_X}\left(1/\sigma_i^2\right)$. Therefore,
\footnotesize\begin{equation}\label{eq:Mn}
M_n\left(\widehat{\lambda}_i\right)\overset{\text{a.s.}}{\longrightarrow}\diag\left(1-\frac{\sigma_1^2}{\sigma_i^2}, \dots, 1-\frac{\sigma_{i-1}^2}{\sigma_i^2}, 0, 1-\frac{\sigma_{i+1}^2}{\sigma_i^2}, \dots, 1-\frac{\sigma_k^2}{\sigma_i^2}\right)
\end{equation}\normalsize
Recall that $U_{n,k}^H\widehat{v}_i$ must be in the kernel of $M_n\left(\widehat{\lambda}_i\right)$. Therefore, any limit point of $U_{n,k}^H\widehat{v}_i$ is in the kernel of the matrix on the right hand side of (\ref{eq:Mn}). Therefore, for $i\neq j$, $i=1,\dots,\widehat{k}$, $j=1,\dots,k$, we must have that $\left(1-\frac{\sigma_j^2}{\sigma_i^2}\right)\langle u_j,\widehat{v}_i\rangle=0$. As $\sigma_i^2\neq\sigma_j^2$, for this condition to be satisfied we must have that for $j\neq i$, $i=1,\dots,\widehat{k}$, $j=1,\dots,k$, $\langle u_j,\widehat{v}_i\rangle\overset{\text{a.s.}}{\longrightarrow}0$.

Recall that our observed vectors $y_i\in\complex^{n\times 1}$ have covariance matrix $U_{n,k}\Sigma U_{n,k}^H+I_n=P_n+I_n$. Therefore, our observation matrix, $Y_n$ which is a $n\times m$ matrix, may be written $Y_n=\left(P_n+I_n\right)^{1/2}Z_n$. The sample covariance matrix, $S_n=\frac{1}{m}Y_nY_n^H$, may be written $S_n=\left(I_n+P_n\right)^{1/2}X_n\left(I_n+P_n\right)^{1/2}$. By similarity transform, if $\widehat{v}_i$ is a unit-norm eigenvector of $\widehat{X}_n$ then $\widehat{s}_i=\left(I_n+P_n\right)^{1/2}\widehat{v}_i$ is an eigenvector of $S_n$. If $\widehat{u}_i=\widehat{s}_i/\|\widehat{s}_i\|$ is a unit-norm eigenvector of $S_n$, it follows that
\begin{equation*}
\langle u_j,\widehat{u}_i\rangle=\frac{\sqrt{\sigma_i^2+1}\langle u_j,\widehat{v}_i\rangle}{\sqrt{\sigma_i^2|\langle u_j,\widehat{v}_i\rangle|^2+1}}
\end{equation*}
As $\langle u_j,\widehat{v}_i\rangle\overset{\text{a.s.}}{\longrightarrow}0$ for all $i\neq j$, $i=1,\dots,\widehat{k}$, $j=1,\dots,k$, it follows that $\langle u_j,\widehat{u}_i\rangle\overset{\text{a.s.}}{\longrightarrow}0$ for all $i\neq j$ $i=1,\dots,\widehat{k}$, $j=1,\dots,k$.

%%%%%%%%%%%%%%%%%%%%% CLAIM %%%%%%%%%%%%%%%%%%%%%%%%%%%%%%%%%%%%%%%%%%%%%

\textit{Claim 5.1:}  We conjecture that this result holds for the general case of $i\neq
j$, $i=1,\dots,\widehat{k}$, $j=1,\dots,k$, not just when $\widehat{k}=\keff=k$. Consider
the case when $k=1$. For $i>2$, if $\widehat{\lambda}_i$ is an eigenvalue of
$\widehat{X}_n=X_n(I_n+\sigma^2uu^H)$, then it satisfies
$\det(\widehat{\lambda}_iI_n-X_n(I_n+\sigma^2uu^H)) =
\det(\widehat{\lambda}_iI_n-X_n)\det(I_n-(\widehat{\lambda}_iI_n-X_n)^{-1}X_n\sigma^2uu^H)=0$. Therefore,
if $\widehat{\lambda}_i$ is not an eigenvalue of $X_n$, the corresponding unit norm
eigenvector $\widehat{v}_i$ is in the kernel of
$I_n-(\widehat{\lambda}_iI_n-X_n)^{-1}X_n\sigma^2uu^H$. Therefore
\begin{equation*}
  |\langle \widehat{v}_i,u\rangle |^2 = \frac{1}{\sigma^4u^HX_n\left(\widehat{\lambda}_iI_n-X_n\right)^{-2}X_nu}.
\end{equation*}
Recall that Weyl's interlacing lemma for eigenvalues gives $\lambda_i(X_n)\leq
\widehat{\lambda}_i\leq \lambda_{i-1}(X_n)$. Letting $X_n=V_n\Lambda_nV_n^H$ and
$w=V_n^Hu$, we see the importance of the
asymptotic spacing of eigenvalues of $X_n$ in
%\begin{equation*}
%  u^HX_n(\widehat{\lambda}_jI_n-X_n)^{-2}X_nu =\sum_{\ell=1}^n\frac{|w_\ell|^2\lambda_\ell^2(X_n)}{\left(\widehat{\lambda}_j-\lambda_\ell\right)^2}\geq\frac{|w_{j-1}|^2\lambda_{j-1}^2(X_n)}{|\lambda_{j-1}-\lambda_j|^2}+\frac{|w_{j}|^2\lambda_j^2(X_n)}{|\lambda_{j-1}-\lambda_j|^2}.
%\end{equation*}
\begin{equation*}
  u^HX_n(\widehat{\lambda}_iI_n-X_n)^{-2}X_nu
  =\sum_{\ell=1}^n\frac{|w_\ell|^2\lambda_\ell^2(X_n)}{\left(\widehat{\lambda}_i-\lambda_\ell\right)^2}\geq
  \frac{\min_j\lambda_j^2(X_n)\min_j|w_j|^2}{\max_j |\lambda_{j-1}-\lambda_j|^2}
\end{equation*}
%In  \cite{jiang2004limiting} it is shown that $\min_j\lambda_j^2(X_n)=\lambda_n^2(X_n)
%\convas (1-\sqrt{c})^4$. The typical spacing between eigenvalues is $O(1/n)$ while the
%typical magnitude of $w_i^2$ is $O(1/n)$. Therefore, the above inequality will typically be $O(n)$ and we get the desired
%result of $|\langle \widehat{v}_j,u\rangle |^2\convas 0$. More generally, it is the
%behavior of the largest eigenvalue gap and the smallest element of $w_i$ that drives this
%convergence. Thus, so long as the eigenvector whose elements are $w_i$ are delocalized
%(having elements of $O(1/\sqrt{n})$ and the smallest gap between $k$ successive
%eigenvalues is at least as large as $O(1/n + \epsilon)$, we may bound the right hand side
%of the above inequality. The claim follows after applying a similarity transform as in the
%proof of Theorem 5.1.

In \cite{silverstein1985smallest} it is shown that $\min_j\lambda_j^2(X_n)=\lambda_n^2(X_n)
\convas (1-\sqrt{c})^2$. The typical spacing between eigenvalues is $O(1/n)$ while the
typical magnitude of $w_j^2$ is $O(1/n)$ \cite{barvinok2005measure}. Therefore, the right hand side of the above inequality will typically be $O(n)$ and we get the desired
result of $|\langle \widehat{v}_i,u\rangle |^2\convas 0$. More generally, it is the
behavior of the largest eigenvalue gap and the smallest element of $w_i$ that drives this
convergence. Thus, so long as the eigenvector whose elements are $w_i$ are delocalized
(i.e. having elements of $O(1/\sqrt{n})$) and the smallest gap between $k$ successive
eigenvalues is at least as large as $O(1/(n^{(0.5+ \epsilon)})$, the right hand side of the inequality will be unbounded with $n$. The claim follows after applying a similarity transform as in the
proof of Theorem 5.1.


\end{proof}
