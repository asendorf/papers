\documentclass[11pt]{article}
\usepackage{times,geometry,algorithm,algorithmic,amsmath,amssymb,amstext,color}
\geometry{letterpaper, tmargin=1in,bmargin=1in,lmargin=1in,rmargin=1in}
\title{Point-by-point comment reply}
\author{Nicholas Asendorf and Raj Rao Nadakuditi}
\date{T-SP-13946-2012.R1 - IEEE Transactions on Signal Processing - ``The Performance of a Matched Subspace Detector that Uses Subspaces Estimated from Finite, Noisy, Training Data"}

% Created by S. Boyd and L. Vandenberghe
% some traditional definitions that can be blamed on craig barratt
\newcommand{\BEAS}{\begin{eqnarray*}}
\newcommand{\EEAS}{\end{eqnarray*}}
\newcommand{\BEA}{\begin{eqnarray}}
\newcommand{\EEA}{\end{eqnarray}}
\newcommand{\BEQ}{\begin{equation}}
\newcommand{\EEQ}{\end{equation}}
\newcommand{\BIT}{\begin{itemize}}
\newcommand{\EIT}{\end{itemize}}

% Detection stuff
\newcommand{\detgtrless}{\overset{H_1}{\underset{H_0}{\gtrless}}}
\newcommand{\convas}{\overset{\textrm{a.s.}}{\longrightarrow}}

%Algorithm Stuff
\renewcommand{\algorithmicrequire}{\textbf{Input:}}
\renewcommand{\algorithmicensure}{\textbf{Output:}}
\newtheorem{prop}{Proposition}[section]
\newtheorem{conj}{Conjecture}[section]
\newtheorem{Th}{Theorem}[section]
\newtheorem{Corr}{Corollary}[section]
\newtheorem{Conj}{Claim}[section]
\newtheorem{Remark}{Remark}[section]

% text abbrevs
\newcommand{\eg}{{\it e.g.}}
\newcommand{\ie}{{\it i.e.}}

\newcommand{\keff}{k_\text{eff}}
% std math stuff
\newcommand{\ones}{\mathbf 1}
\newcommand\reals{\ensuremath{\mathbb{R}}}
\newcommand{\integers}{{\mbox{\bf Z}}}
\newcommand\complex{\ensuremath{\mathbb{C}}}
\newcommand{\symm}{{\mbox{\bf S}}}  % symmetric matrices

% lin alg stuff
\newcommand{\Span}{\mbox{\textrm{span}}}
\newcommand{\range}{{\mathcal R}}
\newcommand{\nullspace}{{\mathcal N}}
\newcommand{\Rank}{\mathop{\bf rank}}
\newcommand{\Tr}{\mathop{\bf tr}}
\newcommand{\cond}{\mathop{\bf cond}}
\newcommand{\diag}{\mathop{\bf diag}}
\newcommand{\lambdamax}{\lambda_{\rm max}}
\newcommand{\lambdamin}{\lambda_{\rm min}}

% probability stuff
\newcommand{\Prob}{\mathop{\bf prob}}
\newcommand{\Expect}{\mathop{\bf E{}}}
\newcommand{\var}{\mathop{\bf var}} % variance
% not sure why we have \Expect and \Prob but \var ???

% convexity & optimization stuff
\newcommand{\Co}{\mathop {\bf conv}} % convex hull
\newcommand{\argmin}{\mathop{\rm argmin}}
\newcommand{\argmax}{\mathop{\rm argmax}}
\newcommand{\epi}{\mathop{\bf epi}}
%\newcommand{\hypo}{\mathop{\bf hypo}}

% sup and inf that look OK in saddle-point form!
%\newcommand{\ourinf}{\mathop{\raisebox{0ex}[0ex][.4ex]{\,inf\,}}}
%\newcommand{\oursup}{\mathop{\raisebox{0ex}[0ex][.4ex]{\,sup\,}}}
\newcommand{\ourinf}{\mathop{\,\mathrm{inf}\, {\rule[-.5ex]{0ex}{0ex}}}}
\newcommand{\oursup}{\mathop{\,\mathrm{sup}\, {\rule[-.5ex]{0ex}{0ex}}}}
%makes latex believe that inf and sup both extend .4ex below
%the baseline

\newcommand{\dist}{\mathop{\bf dist}}
\newcommand{\vol}{\mathop{\bf vol}} % volume
\newcommand{\Card}{\mathop{\bf card}} % cardinality
\newcommand{\sign}{\mathop{\bf sign}}

\newcommand{\dom}{\mathop{\bf dom}} % domain
\newcommand{\aff}{\mathop{\bf aff}} % affine hull
\newcommand{\cl}{\mathop{\bf cl}} % closure
\newcommand{\intr}{\mathop{\bf int}} % interior
\newcommand{\relint}{\mathop{\bf rel int}} % relative interior
\newcommand{\bd}{\mathop{\bf bd}} % boundary

%why do we have the following but not \nust?
\newcommand{\xst}{x^\star}
\newcommand{\lambdast}{\lambda^\star}

% defs for cones & generalized inequalities
% these seem kind of awkward; should fix some day
% rewrite them to use args?
\newcommand{\geqK}{\mathrel{\succeq_K}}
\newcommand{\gK}{\mathrel{\succ_K}}
\newcommand{\leqK}{\mathrel{\preceq_K}}
\newcommand{\lK}{\mathrel{\prec_K}}
\newcommand{\geqKst}{\mathrel{\succeq_{K^*}}}
\newcommand{\gKst}{\mathrel{\succ_{K^*}}}
\newcommand{\leqKst}{\mathrel{\preceq_{K^*}}}
\newcommand{\lKst}{\mathrel{\prec_{K^*}}}
\newcommand{\geqL}{\mathrel{\succeq_L}}
\newcommand{\gL}{\mathrel{\succ_L}}
\newcommand{\leqL}{\mathrel{\preceq_L}}
\newcommand{\lL}{\mathrel{\prec_L}}
\newcommand{\geqLst}{\mathrel{\succeq_{L^*}}}
\newcommand{\gLst}{\mathrel{\succ_{L^*}}}
\newcommand{\leqLst}{\mathrel{\preceq_{L^*}}}
\newcommand{\lLst}{\mathrel{\prec_{L^*}}}

%\newcounter{lecture}
%\newcommand{\lecturefl}[1]{   % use with foiltex landscape
%% \addtocounter{lecture}{1}
% \refstepcounter{lecture}
% \setcounter{equation}{0}
% \setcounter{page}{1}
% \renewcommand{\theequation}{\arabic{equation}}
% \renewcommand{\thepage}{\arabic{lecture}--\arabic{page}}
% \raggedright
% \parindent 0pt
% \rightfooter{\thepage}
% \leftheader{}
% \rightheader{}
% \LogoOff
% \input header
% \begin{center}
%% {\Large \bfseries Lecture \arabic{lecture} \\*[\bigskipamount] {#1}}
%{\Large \bfseries \arabic{lecture}.  {#1}}
% \end{center}
% \MyLogo{#1}
%}

%\newcommand{\lectureflstar}[1]{   % use with foiltex landscape
% \setcounter{equation}{0}
% \setcounter{page}{1}
% \renewcommand{\theequation}{\arabic{equation}}
% \renewcommand{\thepage}{\arabic{page}}
% \raggedright
% \parindent 0pt
% \rightfooter{\thepage}
% \leftheader{}
% \rightheader{}
% \LogoOff
% \input header
% \begin{center}
% {\Large \bfseries #1}
% \end{center}
% \MyLogo{#1}
%}
%\newcounter{oursection}
%\newcommand{\frametitle}[1]{  % for use with foiltex landscape
% \addtocounter{oursection}{1}
%% \setcounter{equation}{0}
% \foilhead[-1.0cm]{#1}
% \LogoOn
%}

\newenvironment{algdesc}%
   {\begin{list}{}{%
    \setlength{\rightmargin}{0\linewidth}%
    \setlength{\leftmargin}{.05\linewidth}}%
    \sffamily\small
    \item[]{\setlength{\parskip}{0ex}\hrulefill\par%
    \nopagebreak{}}}%
   {{\setlength{\parskip}{-1ex}\nopagebreak\par\hrulefill} \end{list}}

\newenvironment{colm}{\left[\begin{array}{c}}{\end{array}\right]}
\newenvironment{colv}{\left(\begin{array}{c}}{\end{array}\right)}


\begin{document}
\maketitle

We thank the reviewers for their decision of publication and for their excellent
comments. We have incorporated the provided comments in the attached manuscript. In
particular the first paragraph in the introduction now highlights previous work in
more depth. We also provided more discussion of the Claim in the appendix, including a
proof sketch to provide more intuition. Because of these additions, for space
considerations, we removed old Figures 6 and 8b. These figures were for the deterministic
setting and showcase the same phenomenon as the stochastic setting. We have also fixed minor typos throughout the
paper and particularly in the bibliography. 

\section*{Reviewer 1}
\begin{enumerate}
\item \textit{Although requirements of my question 5 has not been implemented, I understand this
  could be matter of future work. Hence I consider that all my questions have been
  properly solved}

\textcolor{blue}{Thank you. Comparing these results with a supervised linear discriminant
    would be interesting future work.}
\end{enumerate}

\section*{Reviewer 2}

\begin{enumerate}
\item \textit{There are a few typos in the text that though can be corrected by the
    authors in the preparation of the final version of the manuscript.}

\textcolor{blue}{Thank you for bringing these to our attention. Minor typos and spelling
  mistakes throughout the paper have been corrected.}

\item \textit{The first paragraph of the introduction (from ``Many signal processing [1]''
    to ``signal-bearing training data'' is not completely satisfactory: the contribution
    of the several cited papers should be better indicated; to this end, notice that [3-5]
    refer to processing several returns; [9-10] deal with a second-order model while [3]
    considers a first-order model, etc;)}

\textcolor{blue}{The beginning of this paragraph has been changed to provide more insight
  and showcase the contributions of the cited works.}

\item \textit{In references [1,3,4,6,9,11] L. Scharf should be L.L.Scharf.}

\textcolor{blue}{Thank you for pointing this out. This has been changed.}

\item \textit{In [14] there is a typo in the title: databse should be database}

\textcolor{blue}{Thank you for bringing this to our attention - this has been corrected.}


\end{enumerate}

\section*{Reviewer 3}

No comments provided.

\section*{Reviewer 4}

\begin{enumerate}

\item \textit{The paper is well organized and the covered topic is of relevant interest to
    the community. However, I would recommend the authors to provide somewhat more
    detailed explanation in support of the main Claim, either numerical or
    proofskecth-alike. Indeed, the paper is well written but not completely
    self-consistent, due to the recurrent references to mathematical results in [19] and
    [20], and this has caused my novelty ranking to be a bit lower than the average
    ranking the paper has been provided by me. To be honest, I don't immediately
    understand a direct link between the eigenvalues' spacing characterization and the
    eventual validity of the Claim. If it is claimed, there is no need for an actual
    proof, but nevertheless the hints, where given, should be at least suggesting a raw
    roadmap to be followed in order the link between proven results and conjectured ones
    to be catched. }

\textcolor{blue}{We have provided a more in depth discussion of the Claim in the
  appendix. The discussion provides the fundamentals for the proof of the Claim in the
  rank-1 setting. This roadmap highlights the importance of the spacing of the
  eigenvalues. }

\end{enumerate}

\end{document}
