\documentclass[11pt]{article}
\usepackage{times,amsmath,amssymb,algorithm,algorithmic,geometry,bbm}

\geometry{letterpaper,tmargin=1in,bmargin=1in,lmargin=1in,rmargin=1in}


% Created by S. Boyd and L. Vandenberghe
% some traditional definitions that can be blamed on craig barratt
\newcommand{\BEAS}{\begin{eqnarray*}}
\newcommand{\EEAS}{\end{eqnarray*}}
\newcommand{\BEA}{\begin{eqnarray}}
\newcommand{\EEA}{\end{eqnarray}}
\newcommand{\BEQ}{\begin{equation}}
\newcommand{\EEQ}{\end{equation}}
\newcommand{\BIT}{\begin{itemize}}
\newcommand{\EIT}{\end{itemize}}

% Detection stuff
\newcommand{\detgtrless}{\overset{H_1}{\underset{H_0}{\gtrless}}}
\newcommand{\convas}{\overset{\textrm{a.s.}}{\longrightarrow}}

%Algorithm Stuff
\renewcommand{\algorithmicrequire}{\textbf{Input:}}
\renewcommand{\algorithmicensure}{\textbf{Output:}}
\newtheorem{prop}{Proposition}[section]
\newtheorem{conj}{Conjecture}[section]
\newtheorem{Th}{Theorem}[section]
\newtheorem{Corr}{Corollary}[section]
\newtheorem{Conj}{Claim}[section]
\newtheorem{Remark}{Remark}[section]

% text abbrevs
\newcommand{\eg}{{\it e.g.}}
\newcommand{\ie}{{\it i.e.}}

\newcommand{\keff}{k_\text{eff}}
% std math stuff
\newcommand{\ones}{\mathbf 1}
\newcommand\reals{\ensuremath{\mathbb{R}}}
\newcommand{\integers}{{\mbox{\bf Z}}}
\newcommand\complex{\ensuremath{\mathbb{C}}}
\newcommand{\symm}{{\mbox{\bf S}}}  % symmetric matrices

% lin alg stuff
\newcommand{\Span}{\mbox{\textrm{span}}}
\newcommand{\range}{{\mathcal R}}
\newcommand{\nullspace}{{\mathcal N}}
\newcommand{\Rank}{\mathop{\bf rank}}
\newcommand{\Tr}{\mathop{\bf tr}}
\newcommand{\cond}{\mathop{\bf cond}}
\newcommand{\diag}{\mathop{\bf diag}}
\newcommand{\lambdamax}{\lambda_{\rm max}}
\newcommand{\lambdamin}{\lambda_{\rm min}}

% probability stuff
\newcommand{\Prob}{\mathop{\bf prob}}
\newcommand{\Expect}{\mathop{\bf E{}}}
\newcommand{\var}{\mathop{\bf var}} % variance
% not sure why we have \Expect and \Prob but \var ???

% convexity & optimization stuff
\newcommand{\Co}{\mathop {\bf conv}} % convex hull
\newcommand{\argmin}{\mathop{\rm argmin}}
\newcommand{\argmax}{\mathop{\rm argmax}}
\newcommand{\epi}{\mathop{\bf epi}}
%\newcommand{\hypo}{\mathop{\bf hypo}}

% sup and inf that look OK in saddle-point form!
%\newcommand{\ourinf}{\mathop{\raisebox{0ex}[0ex][.4ex]{\,inf\,}}}
%\newcommand{\oursup}{\mathop{\raisebox{0ex}[0ex][.4ex]{\,sup\,}}}
\newcommand{\ourinf}{\mathop{\,\mathrm{inf}\, {\rule[-.5ex]{0ex}{0ex}}}}
\newcommand{\oursup}{\mathop{\,\mathrm{sup}\, {\rule[-.5ex]{0ex}{0ex}}}}
%makes latex believe that inf and sup both extend .4ex below
%the baseline

\newcommand{\dist}{\mathop{\bf dist}}
\newcommand{\vol}{\mathop{\bf vol}} % volume
\newcommand{\Card}{\mathop{\bf card}} % cardinality
\newcommand{\sign}{\mathop{\bf sign}}

\newcommand{\dom}{\mathop{\bf dom}} % domain
\newcommand{\aff}{\mathop{\bf aff}} % affine hull
\newcommand{\cl}{\mathop{\bf cl}} % closure
\newcommand{\intr}{\mathop{\bf int}} % interior
\newcommand{\relint}{\mathop{\bf rel int}} % relative interior
\newcommand{\bd}{\mathop{\bf bd}} % boundary

%why do we have the following but not \nust?
\newcommand{\xst}{x^\star}
\newcommand{\lambdast}{\lambda^\star}

% defs for cones & generalized inequalities
% these seem kind of awkward; should fix some day
% rewrite them to use args?
\newcommand{\geqK}{\mathrel{\succeq_K}}
\newcommand{\gK}{\mathrel{\succ_K}}
\newcommand{\leqK}{\mathrel{\preceq_K}}
\newcommand{\lK}{\mathrel{\prec_K}}
\newcommand{\geqKst}{\mathrel{\succeq_{K^*}}}
\newcommand{\gKst}{\mathrel{\succ_{K^*}}}
\newcommand{\leqKst}{\mathrel{\preceq_{K^*}}}
\newcommand{\lKst}{\mathrel{\prec_{K^*}}}
\newcommand{\geqL}{\mathrel{\succeq_L}}
\newcommand{\gL}{\mathrel{\succ_L}}
\newcommand{\leqL}{\mathrel{\preceq_L}}
\newcommand{\lL}{\mathrel{\prec_L}}
\newcommand{\geqLst}{\mathrel{\succeq_{L^*}}}
\newcommand{\gLst}{\mathrel{\succ_{L^*}}}
\newcommand{\leqLst}{\mathrel{\preceq_{L^*}}}
\newcommand{\lLst}{\mathrel{\prec_{L^*}}}

%\newcounter{lecture}
%\newcommand{\lecturefl}[1]{   % use with foiltex landscape
%% \addtocounter{lecture}{1}
% \refstepcounter{lecture}
% \setcounter{equation}{0}
% \setcounter{page}{1}
% \renewcommand{\theequation}{\arabic{equation}}
% \renewcommand{\thepage}{\arabic{lecture}--\arabic{page}}
% \raggedright
% \parindent 0pt
% \rightfooter{\thepage}
% \leftheader{}
% \rightheader{}
% \LogoOff
% \input header
% \begin{center}
%% {\Large \bfseries Lecture \arabic{lecture} \\*[\bigskipamount] {#1}}
%{\Large \bfseries \arabic{lecture}.  {#1}}
% \end{center}
% \MyLogo{#1}
%}

%\newcommand{\lectureflstar}[1]{   % use with foiltex landscape
% \setcounter{equation}{0}
% \setcounter{page}{1}
% \renewcommand{\theequation}{\arabic{equation}}
% \renewcommand{\thepage}{\arabic{page}}
% \raggedright
% \parindent 0pt
% \rightfooter{\thepage}
% \leftheader{}
% \rightheader{}
% \LogoOff
% \input header
% \begin{center}
% {\Large \bfseries #1}
% \end{center}
% \MyLogo{#1}
%}
%\newcounter{oursection}
%\newcommand{\frametitle}[1]{  % for use with foiltex landscape
% \addtocounter{oursection}{1}
%% \setcounter{equation}{0}
% \foilhead[-1.0cm]{#1}
% \LogoOn
%}

\newenvironment{algdesc}%
   {\begin{list}{}{%
    \setlength{\rightmargin}{0\linewidth}%
    \setlength{\leftmargin}{.05\linewidth}}%
    \sffamily\small
    \item[]{\setlength{\parskip}{0ex}\hrulefill\par%
    \nopagebreak{}}}%
   {{\setlength{\parskip}{-1ex}\nopagebreak\par\hrulefill} \end{list}}

\newenvironment{colm}{\left[\begin{array}{c}}{\end{array}\right]}
\newenvironment{colv}{\left(\begin{array}{c}}{\end{array}\right)}


\begin{document}

\textit{Claim 5.1:}  We conjecture that this result holds for the general case of $i\neq
j$, $i=1,\dots,\widehat{k}$, $j=1,\dots,k$, not just when $\widehat{k}=\keff=k$. Consider
the case when $k=1$. For $j>2$, if $\widehat{\lambda}_j$ is an eigenvalue of
$\widehat{X}_n=X_n(I_n+\sigma^2uu^H)$, then it satisfies
$\det(\widehat{\lambda}_jI_n-X_n(I_n+\sigma^2uu^H)) =
\det(\widehat{\lambda}_jI_n-X_n)\det(I_n-(\widehat{\lambda}_jI_n-X_n)^{-1}X_n\sigma^2uu^H)=0$. Therefore,
if $\widehat{\lambda}_j$ is not an eigenvalue of $X_n$, the corresponding unit norm
eigenvector $\widehat{v}_j$ is in the kernel of
$I_n-(\widehat{\lambda}_jI_n-X_n)^{-1}X_n\sigma^2uu^H$. Therefore
\begin{equation*}
  |\langle \widehat{v}_j,u\rangle |^2 = \frac{1}{\sigma^4u^HX_n\left(\widehat{\lambda}_jI_n-X_n\right)^{-2}X_nu}.
\end{equation*}
Recall that Weyl's interlacing lemma for eigenvalues gives $\lambda_j(X_n)\leq
\widehat{\lambda}_j\leq \lambda_{j-1}(X_n)$. Letting $X_n=V_n\Lambda_nV_n^H$ and
$w=V_n^Hu$, we see the importance of the
asymptotic spacing of eigenvalues of $X_n$ in
%\begin{equation*}
%  u^HX_n(\widehat{\lambda}_jI_n-X_n)^{-2}X_nu =\sum_{\ell=1}^n\frac{|w_\ell|^2\lambda_\ell^2(X_n)}{\left(\widehat{\lambda}_j-\lambda_\ell\right)^2}\geq\frac{|w_{j-1}|^2\lambda_{j-1}^2(X_n)}{|\lambda_{j-1}-\lambda_j|^2}+\frac{|w_{j}|^2\lambda_j^2(X_n)}{|\lambda_{j-1}-\lambda_j|^2}.
%\end{equation*}
\begin{equation*}
  u^HX_n(\widehat{\lambda}_jI_n-X_n)^{-2}X_nu
  =\sum_{\ell=1}^n\frac{|w_\ell|^2\lambda_\ell^2(X_n)}{\left(\widehat{\lambda}_j-\lambda_\ell\right)^2}\geq
  \frac{\min_j\lambda_j^2(X_n)\min_j|w_j|^2}{\max_j |\lambda_{j-1}-\lambda_j|^2}
\end{equation*}
In  \cite{jiang2004limiting} it is shown that $\min_j\lambda_j^2(X_n)=\lambda_n^2(X_n)
\convas (1-\sqrt{c})^4$. The typical spacing between eigenvalues is $O(1/n)$ while the
typical magnitude of $w_i^2$ is $O(1/n)$. Therefore, the above inequality will typically be $O(n)$ and we get the desired
result of $|\langle \widehat{v}_j,u\rangle |^2\convas 0$. More generally, it is the
behavior of the largest eigenvalue gap and the smallest element of $w_i$ that drives this
convergence. Thus, so long as the eigenvector whose elements are $w_i$ are delocalized
(having elements of $O(1/\sqrt{n})$ and the smallest gap between $k$ successive
eigenvalues is at least as large as $O(1/n + \epsilon)$, we may bound the right hand side
of the above inequality. The claim follows after applying a similarity transform as in the
proof of Theorem 5.1.
 
%As in the above proof, we may apply a similarity transform to make the claim about the
%eigenvectors of the sample covariance matrix. This argument can be extended to the
%multi-rank case to show the importance of $|\lambda_i-\lambda_{i-k}|^2$, that is, the gap
%(or spacing) of $k$ successive eigenvalues of $X_n$. Understanding this term is an important future research area. 
\end{document}