In this paper, we considered a matched subspace detection problem where the low-rank signal subspace is unknown and must be estimated from finite, noisy, signal-bearing training data. We considered both a stochastic and deterministic model for the testing data. The subspace estimate is inaccurate due to finite and noisy training samples and therefore degrades the performance of plug-in detectors compared to an oracle detector. We showed how the ROC performance curve can be derived from the RMT-aided quantification of the subspace estimation accuracy.

Armed with this RMT knowledge, we derived a new RMT detector that only uses the effective number of informative subspace components, $k_\text{eff}$. Plug-in detectors that use the uninformative components will thus incur a performance degradation, relative to the RMT detector. In settings where a practitioner might play-it-safe and set $\widehat{k}> \widehat{k}_{\text{eff}}$, the performance loss in significant (see Figures \ref{fig:stoch_theory_epsilon} and \ref{fig:determ_theory_epsilon} for a demonstration of how much training data such a play-it-safe plug-in detector would need to match the performance of a $\keff$-tuned RMT detector). This highlights the importance of robust techniques \cite{nadakuditi2010fundamental,johnstone2001distribution,el2007tracy} for estimating $k_\text{eff}$ in subspace based detection schemes as opposed to estimating $k$, particularly in the regime where $k_{\text{eff}} < k$.  We showed in Tables \ref{table:summary_stoch} and \ref{table:summary_determ} that the distributions of the test statistics could be expressed as a weighted sum of independent chi-squared random variables. The associated ROC curves can then be computed using a saddlepoint approximation.

The results in this paper can be extended in several directions. We note that the stochastic detector setting assumed normally distributed training and test data. We can extend the analysis to the Gaussian training data but non-Gaussian test vector setting by `integrating-out' the deterministic detector performance curves with respect to the non-Gaussian distribution of the test-vector. Our results relied on characterization of the quantity $\langle u_{j},\widehat{u}_{i}\rangle$.  Thus analogous performance curves can be obtained for any alternate training data models for which this quantity can be analytically quantified. To that end, the results in \cite{benaych2011singular} facilitate such an analysis for a broader class of models including the correlatted Gaussians training data setting. An extension to the missing data setting might follow a similar approach and appears within reach. Aspects related to rate of convergence are open and will be the subject of future work.