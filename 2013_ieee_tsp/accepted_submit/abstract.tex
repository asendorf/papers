We analyze the performance of a matched subspace detector (MSD) where the test signal vector is assumed to reside in an unknown, low-rank $k$ subspace that must be estimated from finite, noisy, signal-bearing training data. Under both a stochastic and deterministic model for the test vector, subspace estimation errors due to limited training data degrade the performance of the standard plug-in detector, relative to that of an oracle detector. To avoid some of this performance loss, we utilize and extend recent results from random matrix theory (RMT) that precisely quantify the quality of the subspace estimate as a function of the eigen-SNR, dimensionality of the system, and the number of training samples. We exploit this knowledge of the subspace estimation accuracy to derive from first-principles a new RMT detector and to characterize the associated ROC performance curves of the RMT and plug-in detectors. Using more than the a critical number of \textit{informative} components, which depends on the training sample size and eigen-SNR parameters of training data, will result in a performance loss that our analysis quantifies in the large system limit.  We validate our asymptotic predictions with simulations on moderately sized systems.
