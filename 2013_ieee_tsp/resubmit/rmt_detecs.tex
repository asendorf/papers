\textcolor{blue}{We saw in Section \ref{sec:std_detecs} that the plug-in detectors rely on the statistic $\widehat{w}=\widehat{U}^Hy$. Instead of deriving the LRT statistic using the conditional distributions of $y$, we will instead use the conditional distributions of $\widehat{w}$; this will reveal the importance of the matrix $\widehat{U}^HU$. The plug-in detectors assume that $\widehat{U}^HU=I_{\widehat{k}}$, however, the analysis in Section \ref{sec:eigvect_aspects} shows that this assumption is incorrect. Knowing the importance of only using $\keff$ subspace components and armed with the asymptotic diagonal approximation of Corrolary \ref{corr:matrix} and the improved signal variance estimates in (\ref{eq:cov}), we are now in position to answer Problem 1 and derive a new RMT detector for both testing settings.}

\subsection{Stochastic RMT Detector}\label{sec:rmt_stoch}

We begin with the stochastic test setting and form the test vector $\widehat{w}=\widehat{U}^Hy$ where $\widehat{U}$ is the subspace estimated from (\ref{eq:param_estims_stoch}) and $y$ is generated from (\ref{eq:stoch_setup}). The LRT statistic using $\widehat{w}$ depends on the conditional distributions under each hypothesis, which by properties of Gaussian random variables are simply
\begin{equation}\label{eq:stoch_distr}
\begin{aligned}
&\widehat{w}|H_0\sim\mathcal{N}\left(0,I_{\widehat{k}}\right)\,\,\,\text{ and }\,\,\,\widehat{w}|H_1\sim\mathcal{N}\left(0, \widehat{U}^HU\Sigma U^H\widehat{U} +I_{\widehat{k}}\right).\\
\end{aligned}
\end{equation}

We immediately see the matrix of interest, $\widehat{U}^HU\Sigma U^H\widehat{U}$. The plug-in detector substitutes $\widehat{U}$ for $U$ and $\widehat{\Sigma}$ for $\Sigma$; this results in $\widehat{w}|H_1\sim\mathcal{N}(0,\widehat{\Sigma} +I_{\widehat{k}})$. However, Corollary 5.1 shows that this is incorrect by providing the asymptotic limit of the covariance matrix in (\ref{eq:stoch_distr}):
\begin{equation}\label{eq:cov mat}
\widehat{U}^HU\Sigma U^H\widehat{U}+I_{\widehat{k}} \convas \diag\left(|\langle u_i,\widehat{u}_i\rangle|^2\sigma_i^2 + 1\right).
\end{equation}
If $\sigma_i^{2}$ were assumed known, this limit would suffice because we could plug in the results in Proposition \ref{th:angles} to get the desired statistic. However, the signal variances are unknown so $\sigma_i^2$ and subsequently $|\langle u_i,\widehat{u}_i\rangle|^2$ must be estimated from data. For the $\widehat{k}_{\text{eff}}$ subspace components estimated from `Algorithm 2' of \cite{nadakuditi2010fundamental}, we form an improved signal variance estimate, $\widehat{\sigma}_{i_\text{rmt}}^2$,  obtained via (\ref{eq:cov}) and use it to estimate $|\langle u_i,\widehat{u}_i\rangle|^2$, denoted by $|\langle u_i,\widehat{u}_i\rangle|^2_\text{rmt}$. Of course, there are correction terms due to finite system size effects, which we ignore, that affect the convergence properties but not the asymptotic form of the detector.

We obtain the RMT detector by computing the LRT statistic using the conditional distributions of (\ref{eq:stoch_distr}). The covariance matrix of $\widehat{w}|H_1$ is computed by substituting  $|\langle u_i,\widehat{u}_i\rangle|^2_\text{rmt}$ and $\widehat{\sigma}_{i_\text{rmt}}^2$ into the diagonal covariance matrix (\ref{eq:cov mat}). After some straightforward algebra we obtain the desired RMT statistic
\begin{equation*}
\Lambda_{\text{rmt}}(\widehat{w})= \sum_{i=1}^{\widehat{k}}\left(\frac{|\langle u_i,\widehat{u}_i\rangle|^2_{\text{rmt}}\widehat{\sigma}_{i_\text{rmt}}^2}{|\langle u_i,\widehat{u}_i\rangle|^2_{\text{rmt}}\widehat{\sigma}_{i_\text{rmt}}^2 + 1}\right)\widehat{w}_i^2.
\end{equation*}
As seen in Proposition \ref{th:angles}, when $i>k_\text{eff}$, $|\langle u_i,\widehat{u}_i\rangle|^2 \convas 0$. The sum on the right hand side (asymptotically) discards the uninformative subspace components. Thus the RMT detector only uses the $\widehat{k}_{\text{eff}}$ informative components given by (\ref{eq:keff}). Consequently, we obtain the test statistic
\begin{equation}\label{eq:optimal_stat_stoch}
\boxed{\Lambda_{\text{rmt}}(\widehat{w})= \sum_{i=1}^{\widehat{k}_{\text{eff}}}\left(\frac{|\langle u_i,\widehat{u}_i\rangle|^2_{\text{rmt}}\widehat{\sigma}_{i_\text{rmt}}^2}{|\langle u_i,\widehat{u}_i\rangle|^2_{\text{rmt}}\widehat{\sigma}_{i_\text{rmt}}^2 + 1}\right)\widehat{w}_i^2}
\end{equation}
and the RMT detector becomes $\Lambda_{\text{rmt}}(\widehat{w}) \detgtrless \gamma_{\text{rmt}}$
%\begin{equation}\label{eq:optimal_class_stoch}
%\Lambda_{\text{rmt}}(\widehat{w}) \detgtrless \gamma_{\text{rmt}},
%\end{equation}
where the threshold $\gamma_{\text{rmt}}$ is chosen in the usual manner. Note that the stochastic RMT detector also takes the form of (\ref{eq:detector_form}). The principal difference between the RMT test statistic in (\ref{eq:optimal_stat_stoch}) and the plug-in test statistic in (\ref{eq:plugin_stat_stoch}) is the role of $\widehat{k}_{\text{eff}}$ in the former. The scaling factors associated with each $\widehat{w}_i^{2}$ for both detectors are about the same; this is why the plug-in detector that uses $\widehat{k}_{\text{eff}}$ components exhibits the same (asymptotic) performance as the RMT detector, which incorporates knowledge of the subspace estimate accuracy. However, our analysis shows that overcompensating and ``playing-it-safe'' by setting $\widehat{k}>\widehat{k}_{\text{eff}}$ can lead to performance loss.



\begin{table*}[ht!]
\centering
\begin{tabular}{clll}\toprule
 Detector & Detector Statistic $\Lambda(\widehat{w})$  & Distribution  of $\Lambda|H_0$ & Distribution of $\Lambda|H_1$\\
\midrule
%Oracle & $ w^H\left[I-\left(\widehat{U}^HU\Sigma U^H\widehat{U}+I\right)^{-1}\right]w$ &  & \\
Plug-in & $\sum_{i=1}^{\widehat{k}}\left(\frac{\widehat{\sigma}_i^2}{\widehat{\sigma}_i^2+1}\right)\widehat{w}_i^2$ & $\sum_{i=1}^{\widehat{k}}\left(\frac{\widehat{\sigma}_i^2}{\widehat{\sigma}_i^2+1}\right)\chi^2_{1i}$ & $\sum_{i=1}^{\widehat{k}}\left(\frac{\widehat{\sigma}_i^2\left(\sigma^2_i|\langle u_i,\widehat{u}_i\rangle|^2+1\right)}{\widehat{\sigma}_i^2+1}\right)\chi^2_{1i}$\\
 RMT & $\sum_{i=1}^{\widehat{k}_{\text{eff}}}\left(\frac{|\langle u_i,\widehat{u}_i\rangle|^2_{\text{rmt}}\widehat{\sigma}_{i_\text{rmt}}^2}{|\langle u_i,\widehat{u}_i\rangle|^2_{\text{rmt}}\widehat{\sigma}_{i_\text{rmt}}^2+1 }\right)\widehat{w}_i^2$ & $\sum_{i=1}^{\widehat{k}_{\text{eff}}}\left(\frac{\widehat{\sigma}_{i_\text{rmt}}^2|\langle u_i,\widehat{u}_i\rangle|^2_{\text{rmt}}}{\widehat{\sigma}_{i_\text{rmt}}^2|\langle u_i,\widehat{u}_i\rangle|^2_{\text{rmt}}+1}\right)\chi^2_{1i}$ & $\sum_{i=1}^{\widehat{k}_{\text{eff}}}\left(\widehat{\sigma}^2_{i_\text{rmt}}|\langle u_i,\widehat{u}_i\rangle|^2_{\text{rmt}}\right)\chi^2_{1i}$\\
\bottomrule
\end{tabular}
\caption{Summary of the plug-in and RMT stochastic MSDs. See Sections \ref{sec:plugin_stoch} and \ref{sec:rmt_stoch} for derivations.}\vskip-0.2cm
\label{table:summary_stoch}
\vspace{-0.1in}
\end{table*}


%Given an observation vector $y$ from (\ref{eq:stoch_setup}), we form the vector $\widehat{w}=\widehat{U}^Hy$ where $\widehat{U}$ is an estimate of the signal subspace. The table summarizes the test statistic associated with each detector when using testing data generated from the stochastic model. The plug-in and RMT detectors have the form of (\ref{eq:detector_form}). In the CFAR setting, the threshold is  set to obtain the desired false alarm probability. Note the appearance of $k_\text{eff}$ in the random matrix theory detector. The associated distribution of each test statistic under $H_0$ and $H_1$ is provided in the last two columns.


\subsection{Deterministic RMT Detector}\label{sec:rmt_detec_determ}

When forming $\widehat{w}$ with $y$ generated from (\ref{eq:determ_setup}), the conditional distributions of $\widehat{w}$ under each hypothesis are $\widehat{w}|H_0\sim\mathcal{N}(0,I_{\widehat{k}})$ and $\widehat{w}|H_1\sim\mathcal{N}(\widehat{U}^HU\Sigma^{1/2} x, I_{\widehat{k}})$. Again, as $x$ is unknown, we use a GLRT. Employing maximum likelihood estimation on $x$ yields the estimate \textcolor{blue}{$\widehat{x}=\left(\Sigma^{1/2} U^H\widehat{U}\widehat{U}^HU\Sigma^{1/2}\right)^{\dagger}\Sigma^{1/2} U^H\widehat{U}\widehat{w}$ where $\dagger$ denotes the Moore-Penrose pseudoinverse}. After simplifying using $\widehat{x}$ and using the natural logarithm operator as a monotonic operation, the GLRT statistic becomes
\begin{equation*}
\textcolor{blue}{\Lambda(\widehat{w}) = \widehat{w}^H\left(\widehat{U}^HU\Sigma^{1/2}\left(\Sigma^{1/2} U^H\widehat{U}\widehat{U}^HU\Sigma^{1/2}\right)^{\dagger}\Sigma^{1/2} U^H\widehat{U}\right)\widehat{w}}.
\end{equation*}

Consider the term $\widehat{U}^HU$. By Proposition \ref{th:angles} and Claim \ref{conj:angles} and by noting that the eigenvectors are unique up to a phase, \textcolor{blue}{we have that $\widehat{U}^HU \convas BA$ where $B$ is a $\widehat{k}\times k$ matrix and $A$ is a $k \times k$ matrix defined as
\small\begin{equation*}
B_{i\ell}:=\begin{cases} b_i=\exp(j\psi_i) & i=\ell \\ 0 & \text{otherwise} \\ \end{cases},\,A_{i\ell}:=\begin{cases} a_i=|\langle u_i,\widehat{u}_i\rangle| & i=\ell \\ 0 & \text{otherwise} \\ \end{cases}.
\end{equation*}\normalsize
For some $\psi_{i}$, $b_i$ denotes the random phase ambiguity in the eigenvector computation (since eigenvectors are unique up to a phase).}

The plug-in detector assumes that $A=B=I_{\widehat{k}}$, that is $b_i=1$ and $|\langle u_i,\widehat{u}_i\rangle|=1$ . However, as seen in Section \ref{sec:rmt}, we have knowledge of $|\langle u_i,\widehat{u}_i\rangle|$ which we may exploit in deriving a new detector. Using the notation just developed, the GLRT statistic may be written as
\begin{equation*}
\textcolor{blue}{\Lambda(\widehat{w})=\widehat{w}^HBA\Sigma^{1/2}(\Sigma^{1/2}A^HB^HBA\Sigma^{1/2})^{\dagger}\Sigma^{1/2}A^HB^H\widehat{w}.}
\end{equation*}
We use (\ref{eq:cov}) and Proposition \ref{th:angles} to estimate $a_i=\sqrt{|\langle u_i,\widehat{u}_i\rangle|^2_{\text{rmt}}}$. Recall that $\widehat{k}_{\text{eff}}$ is an estimate for the number of $\sigma_i^2$ above the phase transition and note that $a_i=0$ when $\sigma_i^2\leq\sqrt{c}$. Incorporating this into the detector, and noting that $A$, $B$, and $\Sigma$ contain only diagonal elements, the GLRT simplifies to
%\begin{equation*}
%\Lambda(\widehat{w})=\widehat{w}^H\left[\begin{array}{cc} I_{\widehat{k}_{\text{eff}}} & 0 \\ 0 & 0_{\widehat{k}-\widehat{k}_{\text{eff}}}\end{array}\right]\widehat{w},
%\end{equation*}
%which may be written
\begin{equation}\label{eq:optimal_stat_determ}
\boxed{\Lambda_{\text{rmt}}(\widehat{w}) = \sum_{i=1}^{\widehat{k}_{\text{eff}}}\widehat{w}_i^2}
\end{equation}
and the deterministic RMT detector is $\Lambda_{\text{rmt}}(\widehat{w}) \detgtrless \gamma_{\text{rmt}}$
%\begin{equation}\label{eq:optimal_class_determ}
%\Lambda_{\text{rmt}}(\widehat{w}) \detgtrless \gamma_{\text{rmt}},
%\end{equation}
where the threshold $\gamma_{\text{rmt}}$ is chosen in the usual manner. This addresses the problem posed in Section \ref{sec:ps_prob2} for the deterministic test vector setting.  We note that this deterministic RMT detector also takes on the form of (\ref{eq:detector_form}). \textcolor{blue}{In fact, in the deterministic setting, the plug-in and RMT detectors are both `energy detectors' and have the same statistic except for the upper bound in the summation.} As in the stochastic setting, the principal difference between the RMT test statistic in (\ref{eq:optimal_stat_determ}) and the plug-in test statistic in (\ref{eq:plugin_stat_determ}) is the role of $\widehat{k}_{\text{eff}}$ in the former. This is also why the plug-in detector that uses $\widehat{k}_{\text{eff}}$ components exhibits the same performance as the RMT detector, which incorporates knowledge of the subspace estimates. 
\begin{table*}[ht!]
\centering
\begin{tabular}{clll}\toprule
 Detector & Detector Statistic $\Lambda(\widehat{w})$  & Distribution of $\Lambda|H_0$ & Distribution of $\Lambda|H_1$\\
\midrule
%Oracle & $ w^H\left(\widehat{U}^HU\left(U^H\widehat{U}\widehat{U}^HU\right)^{-1}U^H\widehat{U}\right)w$ &  & \\
Plug-in & $\sum_{i=1}^{\widehat{k}}\widehat{w}_i^2$ & $\chi^2_{\widehat{k}}$ & $\chi^2_{\widehat{k}}\left(\sum_{i=1}^{\widehat{k}_{\text{eff}}}\sigma_i^2|\langle u_i,\widehat{u}_i\rangle|^2x_i^2\right)$\\
 RMT& $\sum_{i=1}^{\widehat{k}_{\text{eff}}}\widehat{w}_i^2$ & $\chi^2_{\widehat{k}_{\text{eff}}}$ & $\chi^2_{\widehat{k}_{\text{eff}}}\left(\sum_{i=1}^{\widehat{k}_{\text{eff}}}\sigma_i^2|\langle u_i,\widehat{u}_i\rangle|^2x_i^2\right)$\\
\bottomrule
\end{tabular}
\caption{Summary of the plug-in and RMT deterministic MSDs. See Sections \ref{sec:plugin_determ} and \ref{sec:rmt_detec_determ} for derivations.}\vskip-0.2cm
\label{table:summary_determ}
\vspace{-0.3in}
\end{table*}



%Given an observation vector $y$ from (\ref{eq:determ_setup}), we form the vector $\widehat{w}=\widehat{U}^Hy$ where $\widehat{U}$ is an estimate of the signal subspace. The table summarizes the test statistic associated with each detector for the deterministic setting. The plug-in and RMT detectors have the form of (\ref{eq:detector_form}). In the CFAR setting, the threshold is  set to obtain the desired false alarm probability. Note the appearance of $k_\text{eff}$ in the RMT detector. The associated distribution of each test statistic under $H_0$ and $H_1$ is provided in the last two columns. The notation $\chi^2_{k}(\delta)$ is a noncentral chi-square random variable with $k$ degrees of freedom and non-centrality parameter $\delta$.
