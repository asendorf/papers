We now consider the alternative deterministic test vector model (\ref{eq:determ_setup}) and derive a plug-in and RMT detector for this setting. As in the stochastic setting, we work with the processed test vector $w=\widehat{U}^{H}y$. When using the deterministic model, the conditional distributions of the test vector under each hypothesis are simply
\begin{equation*}
\begin{aligned}
&w|H_0\sim\mathcal{N}(0,I_{\widehat{k}})\\
&\textcolor{blue}{w|H_1\sim\mathcal{N}(\widehat{U}^HU\Sigma^{1/2} x, I_{\widehat{k}}).}\\
\end{aligned}
\end{equation*}
However, as $x$ is unknown, we employ the GLRT where $\Lambda(w) = \frac{\max_x f(w|H_1)}{f(w|H_0)}$. The GLRT statistic for our processed data $w$ is
\begin{equation}\label{eq:glrt_determ}
\textcolor{blue}{\Lambda(w)=\frac{\max_x\mathcal{N}(\widehat{U}^HU\Sigma^{1/2} x,I_{\widehat{k}})}{\mathcal{N}(0,I_{\widehat{k}})}.}
\end{equation}
Given a dimension estimate, $\widehat{k}$, $\widehat{U}$ is a $n\times\widehat{k}$ matrix comprised of the top $\widehat{k}$ eigenvectors of the sample covariance matrix of the training data. This is exactly the same estimate as in the stochastic setting. Employing maximum likelihood estimation on $x$ in the GLRT in (\ref{eq:glrt_determ}) yields the estimate \textcolor{blue}{$\widehat{x}=\left(\Sigma^{1/2} U^H\widehat{U}\widehat{U}^HU\Sigma^{1/2}\right)^{\dagger}\Sigma^{1/2} U^H\widehat{U}w$ where $\dagger$ denotes the Moore-Penrose pseudoinverse}. After simplifying using $\widehat{x}$ and using the natural logarithm operator as a monotonic operation, the GLRT statistic becomes
\begin{equation*}
\textcolor{blue}{\Lambda(w) = w^H\left(\widehat{U}^HU\Sigma^{1/2}\left(\Sigma^{1/2} U^H\widehat{U}\widehat{U}^HU\Sigma^{1/2}\right)^{\dagger}\Sigma^{1/2} U^H\widehat{U}\right)w}
\end{equation*}
which simplifies to
\begin{equation}\label{eq:oracle_stat_determ}
\textcolor{blue}{\Lambda(w) = w^H\left(\widehat{U}^HU\left(U^H\widehat{U}\widehat{U}^HU\right)^{\dagger} U^H\widehat{U}\right)w.}
\end{equation}
Notice that $\Sigma$ does not appear in the test statistic.

\subsection{Plug-in Detector}\label{sec:plugin_determ}
The statistic in (\ref{eq:oracle_stat_determ}) is not realizable as $k$, $U$, and $\Sigma$ are unknown. One may then substitute a ML estimate for $U$ in (\ref{eq:oracle_stat_determ}) as in \cite{jin2005cfar} and \cite{mcwhorter2003matched}. The ML estimate (in the large-sample, small matrix setting) for $U$ is the same as in the stochastic setting (see (\ref{eq:param_estims_stoch})) because we have not altered the assumptions on the training data.

By replacing $U$ with the estimate $\widehat{U}$ we obtain the plug-in detector which employs the test statistic
\begin{equation*}
\Lambda_{\text{plugin}}(w)= w^H\left(\widehat{U}^H\widehat{U}\left(\widehat{U}^H\widehat{U}\widehat{U}^H\widehat{U}\right)^{\dagger}\widehat{U}^H\widehat{U}\right)w.\\
\end{equation*}
This simplifies to
\begin{equation}\label{eq:plugin_stat_determ}
\boxed{\Lambda_{\text{plugin}}(w) = w^Hw=\sum_{i=1}^{\widehat{k}}w_i^2}
\end{equation}
and our detector becomes
\begin{equation}\label{eq:plugin_class_determ}
{\Lambda_{\text{plugin}}(w) \detgtrless \gamma_{\text{plugin}}},
\end{equation}
where the threshold $\gamma_{\text{plugin}}$ is chosen in the usual manner. This deterministic plug-in detector is an `energy detector' and also takes the form of (\ref{eq:detector_form}).

The plug-in detector assumes that the estimated signal subspace $\widehat{U}$ is equal to the true signal subspace $U$. We saw in the stochastic setting that we should only choose $\keff$ components. We discuss the (asymptotic) optimality of choosing $\keff$ components for deterministic detectors next.


\subsection{Random Matrix Theory Detector}\label{sec:optimal_determ}

Consider the term $\widehat{U}^HU$. By Corollary \ref{corr:matrix} and by noting that the eigenvectors are unique up to a phase, \textcolor{blue}{we have that $\widehat{U}^HU \convas BA$ where$B$ is a $\widehat{k}\times\min(\widehat{k},k)$ matrix and $A$ is a $\min(\widehat{k},k)\times k$ matrix defined as
\begin{equation*}
B_{i\ell}:=\begin{cases} b_i=\exp(j\psi_i) & i=\ell \\ 0 & \text{otherwise} \\ \end{cases},\,\,\,\,\,A_{i\ell}:=\begin{cases} a_i=|\langle u_i,\widehat{u}_i\rangle| & i=\ell \\ 0 & \text{otherwise} \\ \end{cases}.
\end{equation*}
For some $\psi_{i}$, $b_i$ denotes the random phase ambiguity in the eigenvector computation (since eigenvectors are unique up to a phase).}

The plug-in detector assumes that $A=B=I_{\widehat{k}}$, that is $b_i=1$ and $|\langle u_i,\widehat{u}_i\rangle|=1$ . However, as seen in Section \ref{sec:rmt}, we have knowledge of $|\langle u_i,\widehat{u}_i\rangle|$ which we may exploit in deriving a new detector. Using the notation just developed, the GLRT statistic may be written as
\begin{equation*}
\Lambda(w)=w^HBA(A^HB^HBA)^{\dagger}A^HB^Hw.
\end{equation*}
We use Proposition \ref{th:angles} to estimate $a_i=\sqrt{|\langle u_i,\widehat{u}_i\rangle|^2_{\text{rmt}}}$. Recall that $k_\text{eff}$ is the number of $\sigma_i^2$ above the phase transition and note that $a_i=0$ when $\sigma_i^2\leq\sqrt{c}$. Incorporating this into the detector, and noting that $A$ and $B$ contain only diagonal elements, the GLRT simplifies to
\begin{equation*}
\Lambda(w)=w^H\left[\begin{array}{cc} I_{\min(\widehat{k},k_\text{eff})} & 0 \\ 0 & 0_{\widehat{k}-\min(\widehat{k},k_\text{eff})}\end{array}\right]w.
\end{equation*}
After simplification, the RMT statistic becomes
\begin{equation}\label{eq:optimal_stat_determ}
\boxed{\Lambda_{\text{rmt}}(w) = \sum_{i=1}^{\min(\widehat{k},k_\text{eff})}w_i^2}
\end{equation}
and our detector becomes
\begin{equation}\label{eq:optimal_class_determ}
{\Lambda_{\text{rmt}}(w) \detgtrless \gamma_{\text{rmt}}},
\end{equation}
where the threshold $\gamma_{\text{rmt}}$ is chosen in the usual manner. This addresses the problem posed in Section \ref{sec:ps_prob2} for the deterministic test vector setting.  We note that this deterministic RMT detector also takes on the form of (\ref{eq:detector_form}). \textcolor{blue}{In fact, in the deterministic setting, the plug-in and RMT detectors are both `energy detectrs' and have the same statistic except for the upper bound in the summation.} As in the stochastic setting, the principal difference between the RMT test statistic in (\ref{eq:optimal_stat_determ}) and the plug-in test statistic in (\ref{eq:plugin_stat_determ}) is the role of $\keff$ in the former. This is also why the plug-in detector that uses $\keff$ components exhibits the same performance as the RMT detector, which incorporates knowledge of the subspace estimates. 
\begin{table*}[h]
\centering
\begin{tabular}{clll}\toprule
 Detector & Detector Statistic $\Lambda(w)$  & Distribution of $\Lambda|H_0$ & Distribution of $\Lambda|H_1$\\
\midrule
%Oracle & $ w^H\left(\widehat{U}^HU\left(U^H\widehat{U}\widehat{U}^HU\right)^{-1}U^H\widehat{U}\right)w$ &  & \\
Plug-in & $\sum_{i=1}^{\widehat{k}}w_i^2$ & $\chi^2_{\widehat{k}}$ & $\chi^2_{\widehat{k}}\left(\sum_{i=1}^{\min(\widehat{k},k_\text{eff})}\sigma_i^2|\langle u_i,\widehat{u}_i\rangle|^2x_i^2\right)$\\
 RMT& $\sum_{i=1}^{\min(k_\text{eff},\widehat{k})}w_i^2$ & $\chi^2_{\min(\widehat{k},k_\text{eff})}$ & $\chi^2_{\min(\widehat{k},k_\text{eff})}\left(\sum_{i=1}^{\min(\widehat{k},k_\text{eff})}\sigma_i^2|\langle u_i,\widehat{u}_i\rangle|^2x_i^2\right)$\\
\bottomrule
\end{tabular}
\caption{Given an observation vector $y$ from (\ref{eq:determ_setup}), we form the vector $w=\widehat{U}^Hy$ where $\widehat{U}$ is an estimate of the signal subspace. The table summarizes the test statistic associated with each detector for the deterministic setting. The plug-in and RMT detectors have the form of (\ref{eq:detector_form}). In the CFAR setting, the threshold is  set to obtain the desired false alarm probability. Note the appearance of $k_\text{eff}$ in the RMT detector. The associated distribution of each test statistic under $H_0$ and $H_1$ is provided in the last two columns. The notation $\chi^2_{k}(\delta)$ is a noncentral chi-square random variable with $k$ degrees of freedom and non-centrality parameter $\delta$.}\vskip-0.2cm
\label{table:summary_determ}
\end{table*}
